\font\sixteen=cmbx15
\font\twelve=cmr12
\font\fonteautor=cmbx12
\font\fonteemail=cmtt10
\font\twelvenegit=cmbxti12
\font\twelvebold=cmbx12
\font\trezebold=cmbx13
\font\twelveit=cmsl12
\font\monodoze=cmtt12
\font\it=cmti12
\voffset=0,959994cm % 3,5cm de margem superior e 2,5cm inferior
\parskip=6pt

\def\titulo#1{{\noindent\sixteen\hbox to\hsize{\hfill#1\hfill}}}
\def\autor#1{{\noindent\fonteautor\hbox to\hsize{\hfill#1\hfill}}}
\def\email#1{{\noindent\fonteemail\hbox to\hsize{\hfill#1\hfill}}}
\def\negrito#1{{\twelvebold#1}}
\def\italico#1{{\twelveit#1}}
\def\monoespaco#1{{\monodoze#1}}
\def\iniciocodigo{\lineskip=0pt\parskip=0pt}
\def\fimcodigo{\twelve\parskip=0pt plus 1pt\lineskip=1pt}

\long\def\abstract#1{\parshape 10 0.8cm 13.4cm 0.8cm 13.4cm
0.8cm 13.4cm 0.8cm 13.4cm 0.8cm 13.4cm 0.8cm 13.4cm 0.8cm 13.4cm
0.8cm 13.4cm 0.8cm 13.4cm 0.8cm 13.4cm
\noindent{{\twelvenegit Abstract: }\twelveit #1}}

\def\resumo#1{\parshape  10 0.8cm 13.4cm 0.8cm 13.4cm
0.8cm 13.4cm 0.8cm 13.4cm 0.8cm 13.4cm 0.8cm 13.4cm 0.8cm 13.4cm
0.8cm 13.4cm 0.8cm 13.4cm 0.8cm 13.4cm
\noindent{{\twelvenegit Resumo: }\twelveit #1}}

\def\secao#1{\vskip12pt\noindent{\trezebold#1}\parshape 1 0cm 15cm}
\def\subsecao#1{\vskip12pt\noindent{\twelvebold#1}}
\def\subsubsecao#1{\vskip12pt\noindent{\negrito{#1}}}
\def\referencia#1{\vskip6pt\parshape 5 0cm 15cm 0.5cm 14.5cm 0.5cm 14.5cm
0.5cm 14.5cm 0.5cm 14.5cm {\twelve\noindent#1}}

%@* .

\twelve
\vskip12pt
\titulo{Interface de Usuário Weaver}
\vskip12pt
\autor{Thiago Leucz Astrizi}
\vskip6pt
\email{thiago@@bitbitbit.com.br}
\vskip6pt

\abstract{This article contains the implementation of the user
  interface used by Weaver Game Engine. The code presented here are
  intended to be used when creating buttons, text, menus and other
  user interface elements. It basically manages shaders, create
  elements that can be moved, rotated, clicked and can react to mouse
  hovering. The API presented here are intended to be flexible, the
  user can extend it and change its behavious registering new
  functions.}


\vskip 0.5cm plus 3pt minus 3pt

\resumo{Este artigo contém a implementação da interface de usuário do
  Moto de Jogos Weaver. O código apresentado aqui é projetado para ser
  usado ao criar botões, texto, menus e outros elementos de interface
  de usuário. O código deste artigo basicamente gerencia ``shaders''
  OpenGL e cria elementos que podem ser movidos, rotacionados,
  clicados e podem reagir quando o mouse passa sobre eles. A API
  apresentada aqui é projetada para ser flexível, o usuário pode
  ampliá-la e mudar seu comportamento registrando novas funções.}

\secao{1. Introdução}

Uma interface de usuário é como um programa se comunica com o usuário
e obtém informações dele. Em um programa típico temos menus, botões,
janelas de pop-up e outros elementos de interface típico. Mas em jogos
de cmputador a intrface de usuário tende a ser muito mais simples
conveitualmente, geralmente formada por alguns menus e por elementos
de visualização que dão informações sobre o estado do jogo. POr
exemplo, um número no canto da tela indicando quantas vidas o jogador
tem.

O fato é que no caso de um jogo de computador, não podemos assumir
nada sobre a aparência dos menus, sobre o que eles vã fazer e sobre o
que é cada elemento de interface. Um jogo de estratégia poderá ter
muuitos menus quando alguém clica com o mouse em uma unidade. Um jogo
de plataforma pode ter apenas informações sobre itens equipados e
número de vidas. A aparência de tais elementos pode variar enormemente
dependendo do estilo do jogo.

A única característica universal que iremos assumir para os elementos
de interface com o usuário é que eles sempre devem aparecer sobre o
cenário do jogo, não deve ser obscurecido por objetos que fazem parte
do mundo do jogo. Tipicamente renderizaremos as interfaces de usuário
depois de termos renderizado o mundo de nosso jogo. Além disso, eles
não irão interagir com elementos do mundo do jogo. Eles não fazem
parte diretamente do mundo que o jogo simula, eles são apenas
elementos que dão informações adicionais para o jogador sobre coisas
deste mundo, que de outra forma seriam difíceis de serem expressadas.

O nosso objetivo aqui será criar uma API onde o usuário pode criar uma
nova interface de usuário invocando a função:

\iniciocodigo
@<Declaração de Função (interface.h)@>=
struct user_interface *_Wnew_interface(char *filename, char *shader_filename,
                                      float x, float y, float z, float width,
                                      float height);
@
\fimcodigo

Onde o primeiro argumento é o nome de um arquivo que será aberto e
interpretado descrevendo a textura da interface (pode ser um arquivo
com uma imagem, por exemplo) e o segundo argumento é um nome de
arquivo contendo o shader. Ambos os argumentos podem ser NULL. Um
shader nulo significa que um shader padrão será usado. Um arquivo
vazio significa que nenhuma textura será preenchida mas usaremos o
shader para desenhar a interface na tela. Os demais argumentos são a
posição e tamanho inicial da interface na tela.

Podemos definir seus shaders mais facilmente se fornecermos nossa
própria biblioteca com definições de funções para o código GLSL. Para
isso, podemos passar uma string com código de funções para a função
abaixo:

\iniciocodigo
@<Declaração de Função (interface.h)@>+=
void _Wset_interface_shader_library(char *source);
@
\fimcodigo

Uma vez que tenhamos uma interface, poderemos movê-la com a função:

\iniciocodigo
@<Declaração de Função (interface.h)@>+=
void _Wmove_interface(struct user_interface *i, float x, float y, float z);
@
\fimcodigo

Assumimos que a posição de uma interface é a coordenada em pixels de
seu centro. O eixo $x$ e $y$ representa a posição horizontal e
vertical. O eixo $z$ determina quais interfaces aparecerão na frente
se estiverem ocupando as mesmas posições no espaço.

Podemos também rotacioná-la com a função:

\iniciocodigo
@<Declaração de Função (interface.h)@>+=
void _Wrotate_interface(struct user_interface *i, float rotation);
@
\fimcodigo

A função acima recebe o quanto a imagem será rotacionada em relação à
orientação padrão usando radianos como medida.

Podemos redimencionar a interface com a função abaixo:

\iniciocodigo
@<Declaração de Função (interface.h)@>+=
void _Wresize_interface(struct user_interface *i,
                        float new_width, float new_height);
@
\fimcodigo

A função abaixo renderiza na tela todas as interfaces disponíveis, sem
precisarmos manualmente indicá-las:

\iniciocodigo
@<Declaração de Função (interface.h)@>+=
void _Wrender_interface(unsigned long long time);
@
\fimcodigo

O parâmetro de tempo passado é para que a função tenha noção de quanto
tempo em microsegundos se passaram. Isso é útil para renderizar
corretamente interfaces com texturas animadas.

Mas renderizar todas as interfaces criadas pode não ser o desejado
pelo usuário. Pode ser que queiramos renderizar somente as últimas
interfaces criadas à partir de um ponto da história.  Para isso, a
função abaixo cria uma marcação no nosso histórico de criação de
interfaces. E toda vez que pedimos para renderizar, somente as
interfaces criadas depois da marcação serão rendrizadas:

\iniciocodigo
@<Declaração de Função (interface.h)@>+=
void _Wmark_history_interface(void);
@
\fimcodigo

Mas e se queremos renderizar somente algumas das interfaces criadas
antes da marcação? Neste caso, podemos simplismente criar uma nova
interface que na verdade é uma ligação para uma anterior:

\iniciocodigo
@<Declaração de Função (interface.h)@>+=
struct user_interface *_Wlink_interface(struct user_interface *i);
@
\fimcodigo

Já para interagir com todas as interfaces criadas no passado até a
última marcação, usamos a função abaixo. Ela irá executar as ações
programadas quando um usuário passa o mouse sobre uma interface,
retira o mouse sobre uma interface ou clica nela:

\iniciocodigo
@<Declaração de Função (interface.h)@>+=
void _Winteract_interface(int mouse_x, int mouse_y, bool left_click,
                          bool middle_click, bool right_click);
@
\fimcodigo


Mas como apagar iterfaces uma vez que não precisamos mais delas? Para
isso fornecemos a função abaixo que apaga todas as interfaces criadas
anteriormene até a última marcação na história. Ela também apaga a
última marcação existente, voltando o estado de nossa biblioteca até
como eraimediatamente antes de tal marcação ser criada. As interfaces
qu existirem antes de tal marcação, se existirem, serão novamente
renderizadas.

\iniciocodigo
@<Declaração de Função (interface.h)@>+=
void _Wrestore_history_interface(void);
@
\fimcodigo

Tudo isso vai requerer que gerenciemos nosso histórico de interfaces,
suas marcações e seus shaders. E isso vai requerer que aloquemos e
desaloquemos memória. Há dois tipos de alocações que podemos fazer:
coisas permanentes que ficarão alocadas por possivelmente um bom tempo
e coisas temporárias que serão rapidamente desalocadas. Vamos então
armazenar a função de alocação e desalocação para estes dois casos:

\iniciocodigo
@<Variáveis Locais (interface.c)@>=
static void *(*permanent_alloc)(size_t) = malloc;
static void *(*temporary_alloc)(size_t) = malloc;
static void (*permanent_free)(void *) = free;
static void (*temporary_free)(void *) = free;
@
\fimcodigo

Por padrão, usaremos simplesmente a função de alocação e desalocação
da biblioteca padrão. Mas o usuário pode personalizar cada uma destas
funções. Além delas, vamos também estabelecer funções personalizáveis
que serão executadas imediatamente antes e depois que formos carregar
uma nova interface. Inicialmente tais funções serão nulas, mas elas
podem ser depois ajustadas pelo usuário:

\iniciocodigo
@<Variáveis Locais (interface.c)@>=
static void (*before_loading_interface)(void) = NULL;
static void (*after_loading_interface)(void) = NULL;
@
\fimcodigo

A ideia é que todas as funções persnalizáveis serão definidas durante
a inicialização da nossa API:

\iniciocodigo
@<Declaração de Função (interface.h)@>+=
#include <stdlib.h> // Define tipo 'size_t'
void _Winit_interface(int *window_width, int *window_height,
                      void *(*permanent_alloc)(size_t),
                      void (*permanent_free)(void *),
                      void *(*temporary_alloc)(size_t),
                      void (*temporary_free)(void *),
                      void (*before_loading_interface)(void),
                      void (*after_loading_interface)(void),
                      ...);
@
\fimcodigo


É possível inicializar as funções de desalocação como
\monoespaco{NULL}. Isso significa que nós não iremos desalocar o que
foi alocado. Isso pode ser útil caso seu gerenciador de memória
gerencie de alguma forma a coleta de lixo e não quer ter interferência
no processo.

A função acima também aceita receber um número variável de
argumentos. Primeiro dois ponteiros indicando onde podemos consultar
de forma atualizada a largura e altura de nossa janela. Eles serão
armazenados aqui:

\iniciocodigo
@<Variáveis Locais (interface.c)@>=
static int *window_width = NULL, *window_height = NULL;
@
\fimcodigo

Depois, as seis funções personalizáveis vistas acima. Depois delas, os
argumentos adicionais serão uma lista terminada em NULL de uma string
seguida de uma função geradora de interfaces. A string representa uma
extensão (por exemplo ``gif'', ``jpg'' ou outras) e a função que a
sucede é representada pela seguinte definição de tipo:

\iniciocodigo
@<Macros Locais (interface.c)@>=
typedef void pointer_to_interface_function(void *(*)(size_t), void (*)(void *),
                                          void *(*)(size_t), void (*)(void *),
                                          void (*)(void), void (*)(void),
                                          char *, struct user_interface *);
@
\fimcodigo

Ela recebe como argumento as funções de alocação e desalocação, um
nome de arquivo e um ponteiro para a interface que deve ser atualizada
e preenchida de acordo com o conteúdo do arquivo. Espera-se que a
função consiga abrir e interpretar o arquivo e gerar uma nova
interface à partir dele.  Desta forma, deixamos para o usuário a
responsabilidade de fornecer as funções que criam interfaces à partir
de arquivos de diferentes formatos.

Como apresentamos uma função de inicialização, vamos precisar também
da função de finalização:

\iniciocodigo
@<Declaração de Função (interface.h)@>+=
void _Wfinish_interface(void);
@
\fimcodigo

E isso termina a nossa descrição de todas as funções que
suportaremos.

Além das funções, o comportamento da API poder ser mudado pela
macro \monoespaco{W\_FORCE\_LANDSCAPE} se ela estiver definida. Se
estivermos em um ambiente em que a largura da nossa janela for maior
que a altura, a existência desta macro não fará nenhuma
diferença. Contudo, se a altura for maior que a largura e a macro
estiver definida, então os eixos $x$ e $y$ serão rotacionados 90 graus
no sentido anti-horário. Isso pode ser útil para criar interfaces de
usuário mais consistentes em ambientes móveis, como celulares. Tais
dispositivos podem ser facilmente movidos, então podemos sempre
ajustar para termos mais espaço horizontal que na vertical sabendo que
não é um grande inconveniente para o usuário ter que rotacionar o
dispositivo.

\subsecao{1.1. Programação Literária}

Nossa API será escrita usando a técnica de Programação Literária,
proposta por Knuth em [KNUTH, 1984]. Ela consiste em escrever um
programa de computador explicando didaticamente em texto o que se está
fazendo à medida que apresenta o código. Depois, o programa é
compilado através de programas que extraem o código diretamente do
texto didático. O código deve assim ser apresentado da forma que for
mais adequada para a explicação no texto, não como for mais adequado
para o computador.

Seguindo esta técnica, este documento não é uma simples documentação
do nosso código. Ele é por si só o código. A parte que será extraída e
compilada posteriormente pode ser identificada como sendo o código
presente em fundo cinza. Geralmente começamos cada trecho de código
com um título que a nomeia. Por exemplo, imediatamente antes desta
subseção nós apresentamos uma série de declarações. E como pode-se
deduzir pelo título delas, a maioria será posteriormente posicionada
dentro de um arquivo chamado \monoespaco{interface.h}.

Podemos apresentar aqui a estrutura do arquivo
\monoespaco{interface.h}:

\iniciocodigo
@(src/interface.h@>=
#ifndef __WEAVER_INTERFACE
#define __WEAVER_INTERFACE
#ifdef __cplusplus
extern "C" {
#endif
#include <stdbool.h> // Define tipo 'bool'
#if !defined(_WIN32)
#include <sys/param.h> // Necessário no BSD, mas causa problema no Windows
#endif
@<Inclui Cabeçalhos Gerais (interface.h)@>
@<Macros Gerais (interface.h)@>
@<Estrutura de Dados (interface.h)@>
@<Declaração de Função (interface.h)@>
#ifdef __cplusplus
}
#endif
#endif
@
\fimcodigo

O código acima mostra a burocracia padrão para definir um cabeçalho
para nossa API em C. As duas primeiras linhas mais a última são macros
que garantem que esse cabeçalho não será inserido mais de uma vez em
uma mesma unidade de compilação. As linhas 3, 4, 5, assim como a
penúltima, antepenúltima e a antes da antepenúltima tornam o cabeçalho
adequado a ser inserido em código C++. Essas linhas apenas avisam que
o que definirmos ali deve ser encarado como código C. Por isso o
compilador está livre para fazer otimizações sabendo que não usaremos
recursos da linguagem C++, como sobrecarga de operadores. Logo em
seguida, inserimos um cabeçalho que nos permite declarar o tipo
booleano. E tem também uma parte em vermelha. Note que uma delas é
``Declaração de Função (interface.h)'', o mesmo nome apresentado no trecho de
código mostrado quando descrevemos nossa API antes dessa
subseção. Isso significa que aquele código visto antes será depois
inserido ali. As outras partes em vermelho representam código que
ainda iremos definir nas seções seguintes.

Caso queira observar o que irá no arquivo \monoespaco{interface.c}
associado a este cabeçaho, o código será este:

\iniciocodigo
@(src/interface.c@>=
#include "interface.h"
@<Cabeçalhos Locais (interface.c)@>
@<Macros Locais (interface.c)@>
@<Estrutura de Dados Locais (interface.c)@>
@<Variáveis Locais (interface.c)@>
@<Funções Auxiliares Locais (interface.c)@>
@<Definição de Funções da API (interface.c)@>
@
\fimcodigo

Todo o código que definiremos e explicaremos a seguir será posicionado
nestes dois arquivos. Além deles, nenhum outro arquivo será criado.

\subsecao{1.2. Suportando Múltiplas Threads}

A maior parte do código a ser definido neste documento é portável. O
único requisito assumido é que um contexto OpenGL está ativo e
funcionando. Contudo, há uma parte não-portável que deve ser definida
dependendo do sistema em que estamos: o suporte a mutex.

Um mutex é uma estrutura de dados abstrata usada para controlar acesso
de múltiplos processos a um recurso em comum. Eles serão tratados de
forma diferente dependendo do sistema operacional e ambiente. Devido à
seu caráter não-portável, eles serão introduzidos aqui, separados do
restante do código.

No Linux e BSD, o Mutex é definido pela biblioteca
\monoespaco{pthread} e segue a nomenclatura típica dela. No Windows um
Mutex é chamado de ``seção critica''. No Web Assembly não usaremos
Mutex, pois o cóigo não usará múltiplas threads.

\iniciocodigo
@<Macros Gerais (interface.h)@>=
#if defined(__linux__) || defined(BSD)
#define _MUTEX_DECLARATION(mutex) pthread_mutex_t mutex
#define _STATIC_MUTEX_DECLARATION(mutex) static pthread_mutex_t mutex
#elif defined(_WIN32)
#define _MUTEX_DECLARATION(mutex) CRITICAL_SECTION mutex
#define _STATIC_MUTEX_DECLARATION(mutex) static CRITICAL_SECTION mutex
#elif defined(__EMSCRIPTEN__)
#define _MUTEX_DECLARATION(mutex)
#define _STATIC_MUTEX_DECLARATION(mutex)
#endif
@
\fimcodigo

Isso significa que no Linux e BSD nós precisamos inserir o cabeçalho
da biblioteca \monoespaco{pthread}. No Windows, basta inserir o
cabeçalho padrão do windows.

\iniciocodigo
@<Inclui Cabeçalhos Gerais (interface.h)@>=
#if defined(__linux__) || defined(BSD)
#include <pthread.h>
#elif defined(_WIN32)
#include <windows.h>
#endif
@
\fimcodigo

No nosso código vamos precisar inicializar cada Mutex que
declararmos. Para isso, usaremos a seguinte macro:

\iniciocodigo
@<Macros Locais (interface.c)@>=
#if defined(__linux__) || defined(BSD)
#define MUTEX_INIT(mutex) pthread_mutex_init(mutex, NULL);
#elif defined(_WIN32)
#define MUTEX_INIT(mutex) InitializeCriticalSection(mutex);
#elif defined(__EMSCRIPTEN__)
#define MUTEX_INIT(mutex)
#endif
@
\fimcodigo

Para finalizar um Mutex quando não precisarmos mais, usamos a seguinte
macro:

\iniciocodigo
@<Macros Locais (interface.c)@>=
#if defined(__linux__) || defined(BSD)
#define MUTEX_DESTROY(mutex) pthread_mutex_destroy(mutex);
#elif defined(_WIN32)
#define MUTEX_DESTROY(mutex) DeleteCriticalSection(mutex);
#elif defined(__EMSCRIPTEN__)
#define MUTEX_DESTROY(mutex)
#endif
@
\fimcodigo

Tendo um Mutex, há duas operações que podemos fazer com ele. A
primeira é requerer o uso do Mutex. Neste momento, se algum outro
processo está usando ele, iremos esperar até que o Mutex esteja livre
novamente:

\iniciocodigo
@<Macros Locais (interface.c)@>=
#if defined(__linux__) || defined(BSD)
#define MUTEX_WAIT(mutex) pthread_mutex_lock(mutex);
#elif defined(_WIN32)
#define MUTEX_WAIT(mutex) EnterCriticalSection(mutex);
#elif defined(__EMSCRIPTEN__)
#define MUTEX_WAIT(mutex)
#endif
@
\fimcodigo

E finalmente, depois de deixarmos de usar o recurso guardado pelo
Mutex, podemos liberar ele com o código abaixo:

\iniciocodigo
@<Macros Locais (interface.c)@>=
#if defined(__linux__) || defined(BSD)
#define MUTEX_SIGNAL(mutex) pthread_mutex_unlock(mutex);
#elif defined(_WIN32)
#define MUTEX_SIGNAL(mutex) LeaveCriticalSection(mutex);
#elif defined(__EMSCRIPTEN__)
#define MUTEX_SIGNAL(mutex)
#endif
@
\fimcodigo

\secao{2. Estruturas de Dados}

Nesta seção descreveremos as quatro principais estruturas de dados que
gerenciaremos nesta API: os shaders, as interfaces de usuário, as
ligações para interfaces já criadas e as marcações de tempo.

\subsecao{2.1. Estrutura de Dados de Shader}

Para renderizarmos qualquer coisa na placa de vídeo, é necessário que
sejam definidas as regas para a renderização. Em que posição iremos
colocar o objeto renderizado? Qual a cor que ele terá? Qual a textura?
A renderização irá mudar com o tempo ou dependendo do ângulo de
visualização? Para responder isso, usam-se programas especiais que são
compilados e executados na placa de vídeo. Tais programas são os
shaders.

Mas além do programa em si, para cada shader precisamos armazenar
algumas variáveis adicionais. Shaders possuem suas próprias variáveis
que podem ser modificadas e ajustadas antes deles serem
executados. Para cada variável modificável, temos que armazenar o
endereço no programa de tais variáveis. Só com esta informação podemos
ajustá-las.

Um shader então terá a seguinte forma:

\iniciocodigo
@<Estrutura de Dados Locais (interface.c)@>=
struct shader {
  int type;
  void *next; // Ponteiro de lista encadeada
  GLuint program; // O programa em si
  /* Aqui começam as variáveis modificáveis de shader */
  GLint attribute_vertex_position; // Posição de cada vértice
  GLint uniform_foreground_color, uniform_background_color; // Cor de frente/fundo
  GLint uniform_model_view_matrix; // Matriz com tamanho, rotação e translação
  GLint uniform_interface_size; // Tamanho em píxels do objeto renderizado
  GLint uniform_mouse_coordinate; // Coordenadas do mouse (referencial: interface)
  GLint uniform_time; // Contador de tempo para texturas animadas
  GLint uniform_integer; // Inteiro arbitrário que pode ser passado
  GLint uniform_texture1; // A textura do objeto a ser renderizado
};
@
\fimcodigo

Como usamos declarações do OpenGL, como \monoespaco{GLuint}, vamos
inserir o cabeçalho OpenGL:

\iniciocodigo
@<Inclui Cabeçalhos Gerais (interface.h)@>+=
#if defined(__linux__) || defined(BSD) || defined(__EMSCRIPTEN__)
#include <EGL/egl.h>
#include <GLES3/gl3.h>
#endif
#if defined(_WIN32)
#pragma comment(lib, "Opengl32.lib")
#include <windows.h>
#include <GL/gl.h>
#endif
@
\fimcodigo


As duas primeiras variáveis na estrutura de dados existem porque
iremos armazenar os shaders, assim como outros tipos de estruturas de
dados em uma lista encadeada. O tipo serve para identificar o quê
exatamente está em cada posição da lista e o ponteiro para o próximo
elemento é o que monta a nossa lista encadeada ao ligar cada elemento
ao seguinte.

No caso de uma estrutura de shader, o tipo, primeira variável, terá
sempre o valor de \monoespaco{TYPE\_SHADER}. Todos os tipos diferentes
que podem ser armazenados na lista encadeada são:

\iniciocodigo
@<Macros Locais (interface.c)@>=
#define TYPE_INTERFACE 1 // Uma interface, ou seja, objeto a ser renderizado
#define TYPE_LINK      2 // Ligação para interface anterior
#define TYPE_MARKING   3 // Marcação no histórico de interfaces criadas
#define TYPE_SHADER    4 // Shader para renderizar interface
@
\fimcodigo

A variável \monoespaco{program} é a representação do programa de
shader compilado. E tudo o que vem depois são variáveis indicando onde
cada uma das diferentes variáveis configuráveis de nosso programa de
shader pode ser encontrada. Nem todas as variáveis que aparecem
necessariamente serão usadas por todos os shaders, mas estas são as
variáveis que nossa API suporta. Há variáveis com informações sobre o
vértices a serem renderizado, tamanho, posição, cores, textura e
algumas coisas mais.

\subsecao{2.2. A Estrutura da Interface}

A principal estrutura de dados deste documento é a que armazena
informações sobre a interface de usuário. A descrição dela é:

\iniciocodigo
@<Estrutura de Dados (interface.h)@>=
struct user_interface{
  int type;
  void *next; // Ponteiro de lista encadeada
  float x, y, _x, _y, z;
  float rotation, _rotation;
  float mouse_x, mouse_y;
  GLfloat _transform_matrix[16];
  float height, width;
  float background_color[4], foreground_color[4];
  int integer;
  bool visible;
  struct shader *shader_program;
  _MUTEX_DECLARATION(mutex);
  /* Funções de interação*/
  bool _mouse_over;
  void (*on_mouse_over)(struct user_interface *);
  void (*on_mouse_out)(struct user_interface *);
  void (*on_mouse_left_down)(struct user_interface *);
  void (*on_mouse_left_up)(struct user_interface *);
  void (*on_mouse_middle_down)(struct user_interface *);
  void (*on_mouse_middle_up)(struct user_interface *);
  void (*on_mouse_right_down)(struct user_interface *);
  void (*on_mouse_right_up)(struct user_interface *);
  /* Atributos abaixo devem ser preenchidos por função de carregamento: */
  GLuint *_texture1;
  bool _loaded_texture;
  bool animate;
  unsigned number_of_frames;
  unsigned current_frame;
  unsigned *frame_duration;
  unsigned long _t;
  int max_repetition;
};
@
\fimcodigo

Vamos agora descrever o que significa cada elemento desta estrutura. O
tipo e o ponteiro para o próximo elemento já foram descritos e são os
mesmos que para a estrutura de dados de shaders.

Depois destas duas variáveis, armazenamos as informações sobre posição
e tamanho da interface. Temos a sua posição em pixels (eixo $x$ e
$y$), seu ``índice-z'' (eixo $z$) que determina a ordem em que as
interfaces serão desenhadas e quais aparecerão na frente das outras, a
rotação (em radianos) da interface e a altura e largura (em
pixels). Todos estes valores podem ser lidos pelo usuário, mas não
devem ser modificados diretamente sem usar as funções adequadas para
isso e que serão definidas nas próximas seções. Isso porque para
corretamente mudar a posição, tamanho e rotação da interface,
precisamos atualizar a sua matriz de transfrmação (logo abaixo), que é
como o OpenGL e nossos shaders realmente interpretam tais
informações. E com relação ao eixo $z$, se atualizamos ele, também
temos que atualizar uma outra lista a ser definida que controla a
ordem em que as interfaces são desenhadas na tela.

Outra coisa que deve ser notada é que na verdade temos duas variáveis
para armazenar a posição $x$, duas para armazenar a posição $y$ e duas
para a rotação. Isso ocorre porque quando a
macro \monoespaco{W\_FORCE\_LANDSCAPE} está definida e temos uma tela
com mais altura e largura, assumimos que o nosso eixo $x$ e $y$
estarão rotacionados, bem como a interface que também terá uma rotação
de 90 graus. Nestes casos, as
variáveis \monoespaco{\_x}, \monoespaco{\_y} e \monoespaco{\_rotation}
armazenarão os valores verdadeiros, quando os eixos não estão trocados
e as variáveis \monoespaco{x}, \monoespaco{y} e \monoespaco{rotation}
armazenarão as variáveis com os eixos trocados. Nos outros casos, se a
largura for maior que a altura ou se a macro não estiver definida,
ambos os tipos de variáveis terão sempre o mesmo valor.

Os atributos \monoespaco{mouse\_x} e \monoespaco{mouse\_y} são
reservados para armazenar a coordenada do mouse usando como
referencial não o canto interior esquerdo da tela, mas o canto
inferior esquerdo da interface.

O atributo de cor de frente e de fundo representa cores no formato
RGBA onde cada valor individua é entre 0 e 1. Basicamente tais cores
serão sempre passadas para o shader de renderização. Mas não
necessariamente o shader irá usar tal informação.

Da mesma forma, o atributo inteiro que vem logo em seguida também é um
valor inteiro a ser passado para o shader. Não necessariamente ele
será usado, isso irá depender do shader.

O próximo atributo indica se a interface está visível ou não. Se não
estiver, ela não será renderizada.

Em seguida há um ponteiro para a estrutura de dados de shader
associada à nossa interface. Que mostra como ela será renderizada.

E por fim, é inserida uma declaração de um mutex para que múltiplos
processos possam manipular simultaneamente nossa interface.

Em seguida temos as funções de interação que se forem diferentes
de \monoespaco{NULL}, serão executadas quando o mouse passa sobre a
interface, quando deixa a interface ou quando algum botão do mouse é
pressionado ou solto sobre elas. E as interfaces tem também uma
variável booleana que indica se o mouse está sobre elas ou não.

Os atributos que vem à seguir desta parte inicialmente sempre começam
como sendo 0, falso ou nulo. É de responsabilidade da função de
carregamento preencher tais valores enquanto executa. A função de
carregamento é aquela que é responsável por ler um arquivo de
determinada extensão e gerar corretamente a textura da
interface.Funções de carregamento são informadas na inicialização
desta API.

Primeiro temos um ponteiro para a textura OpenGL que é passada para o
shader. Pode não ter somente uma textura, mas várias. Quando por
exemplo temos uma interface obtida através de um GIF animado, por
exemplo. Cada frame de animação será uma textura diferente.

Caso exista mais de uma textura, a próxima variável booleana determina
se a interface deve ser animada ou não. A variável pode ser modificada
à vontade durante a execução do programa para fazer com que a animação
pause ou continue.

Os dois próximos atributos representam o número total de frames da
animação (será sempre 1 quando não for uma interface animada) e qual o
frame atual em que estamos (o primeiro frame é o 0).

Em seguida encontramos um ponteiro que apontará para um vetor alocado
que conterá a duração de cada frame em unidades de tempo. Se a
interface possuir somente um ou nenhum frame, este ponteiro poderá
estar apontando para NULL.

A variável \monoespaco{\_t} será usada para realizar a contagem de
tempo internamente no caso de termos uma interface animada. Deve
iniciar como zero quando a interface é inicializada e sua textura
termina de ser carregada. Depois seguiremos incrementando o contador
automaticamente para assim sabermos quando o frame atual precisa ser
modificado ou não, a depender da duração de cada frame.

Por fim, o último atributo é, para o caso de termos uma interface
animada, quantas vezes devemos repetir até o final sua animação. O
valor de 0 significa que ela deverá ser repetida para sempre. Um valor
positivo coloca u limite no número de repetições. Quando repetirmos a
animação pela última vez permitida, ela ficará estacionada no último
frame sem voltar novamente para o primeiro.

\subsecao{2.3. Marcações no Histórico}

Como mencionado na Introdução, a lógica de quais interfaces estarão
acessíveis em um dado momento terá relação com as marcações feitas no
nosso histórico de interfaces. Todas as interfaces criadas após a
última marcação estão acessíveis, mas aquelas que forem mais antigas
que isso não poderão ser acessadas. Elas não serão renderizadas e
nem pode-se interagir com elas. Podem ser criadas várias marcações. E
a marcação mais recente existente pode ser destruída, o que destrói
as interfaces criadas depois dela.

Para que isso seja possível, cada marcação precisa memorizar qual era
a marcação anterior a também um ponteiro para o elemento anterior na
lista encadeada. Desta forma, quando apagamos uma marcação, podemos
substituir a marcação memorizada colocando-a no logar daquela que foi
removida e fazer as atualizações devidas de modo mais fácil. Além
disso, faremos cada marcação saber também a quantidade de interfaces
que foi criada depois dela. É o número de interfaces que estão ativas
e existirão enquanto a marcação existir.

\iniciocodigo
@<Estrutura de Dados Locais (interface.c)@>+=
struct marking {
  int type;
  void *next; // Ponteiro para próximo na lista encadeada
  void *prev; // Ponteiro para anterior na lista encadeada
  struct marking *previous_marking;
  unsigned number_of_interfaces;
};
@
\fimcodigo

Note que declaramos essa estrutura de dados internamente no arquivo
\monoespaco{interface.c}. Isso porque essa é uma estrutura que será
útil somente internamente à nossa API. Não haverá motivos para que o
usuário da API precise obter uma marcação destas. Ele irá interagir
com a existência delas somente por meio das funções
\monoespaco{\_Wmark\_history\_interface} e
\monoespaco{\_Wrestore\_history\_interface} descritas na Introdução.

\subsecao{2.4. Ligações para Outras Interfaces}

Caso uma interface esteja inacessível por ser mais antiga que a última
marcação, é possível criar uma ligação para ela com a função
\monoespaco{\_Wlink\_interface}. A ligação para a interface antiga
conta como se fosse uma nova interfae recém-criada e desta forma a
interface antiga volta a estar acessível. Criar uma nova ligação
significa criar e colocar na lista encadeada a seguinte estrutura:

\iniciocodigo
@<Estrutura de Dados Locais (interface.c)@>=
struct link {
  int type;
  void *next; // Ponteiro de lista encadeada
  struct user_interface *linked_interface;
};
@
\fimcodigo

Além das informações necessárias para a lista encadeada como o tipo
que identifica a estrutura e o ponteiro que apontará para o elemento
da próxima posiçao, a única informação que precisamos armazenar nesta
estrutura é um ponteiro para a interface apontada.

\secao{3. Inicialização e Finalização da API}

O objetivo da inicialização desta API é ajustar todas as funções
personalizadas que serão usadas. Já a finalização desaloca as
estruturas alocadas necessárias para armazenar algumas destas funções
e também ajusta tais funções para valores padrão.

\subsecao{3.1. Inicialização}

Aqui nosso objetivo é ajustar as seis funções personalizáveis
discutidas na Introdução e também receber uma lista de tamanho
variável de funções que irão interpretar arquivos com uma dada
extensão.

Para armazenar cada uma destas funções que vai na lista de tamanho
variável, precisamos da seguinte estrutura:

\iniciocodigo
@<Estrutura de Dados Locais (interface.c)@>=
struct file_function {
  char *extension;
  void (*load_texture)(void *(*permanent_alloc)(size_t),
                      void (*permanent_free)(void *),
                      void *(*temporary_alloc)(size_t),
                      void (*temporary_free)(void *),
                      void (*before_loading_interface)(void),
                      void (*after_loading_interface)(void),
                      char *source_filename, struct user_interface *target);
};
static unsigned number_of_file_functions_in_the_list = 0;
static struct file_function *list_of_file_functions = NULL;
@
\fimcodigo

A estrutura basicamente é um par formado por uma função que extrai
texturas de um arquivo (e é informada sobre todas as funções
personalizadas que devem ser usadas) e por uma extensão de arquivo que
ela interpreta. Depois de definir a estrutura, criamos um ponteiro
para ela que irá conter uma lista de funções deste tipo conhecidas. O
ponteiro começa como uma lista vazia.

Depois que esta lista não estiver mais vazia, isto é, depois da
inicialização, nós poderemos percorrê-la para extrair a função correta
dada uma extensão por meio da função auxiliar:

\iniciocodigo
@<Funções Auxiliares Locais (interface.c)@>=
static inline void (*get_loading_function(char *ext))
                          (void *(*permanent_alloc)(size_t),
                           void (*permanent_free)(void *),
                           void *(*temporary_alloc)(size_t),
                           void (*temporary_free)(void *),
                           void (*before_loading_interface)(void),
                           void (*after_loading_interface)(void),
                           char *source_filename, struct user_interface *target){
  unsigned i;
  for(i = 0; i < number_of_file_functions_in_the_list; i ++){
    if(!strcmp(list_of_file_functions[i].extension, ext)){
      return list_of_file_functions[i].load_texture;
    }
  }
  return NULL;
}
@
\fimcodigo

O código acima é verboso porque é de uma função que retorna um
ponteiro para outra função que tem muitos parâmetros. Mas a função
acima na verdade tem um único parâmetro: a extensão que deve ser
procurada na lista.

Como usamos a função de comparar strings, vamos inserir o cabeçalho
padrão com funções envolvendo strings:

\iniciocodigo
@<Cabeçalhos Locais (interface.c)@>=
#include <string.h>
@
\fimcodigo

Podemos agora definir a função de inicialização. O que esta função
fará será preencher as seis funções personalizáveis básicas, contar
quantas outras funções temos que irão interpretar arquivos, alocar o
espaço necessário em nossa lista acima (usando as funções de alocação
informadas) e preencher a lista recém-alocada:

\iniciocodigo
@<Definição de Funções da API (interface.c)@>=
void _Winit_interface(int *window_width_p, int *window_height_p,
                      void *(*new_permanent_alloc)(size_t),
                      void (*new_permanent_free)(void *),
                      void *(*new_temporary_alloc)(size_t),
                      void (*new_temporary_free)(void *),
                      void (*new_before_loading_interface)(void),
                      void (*new_after_loading_interface)(void), ...){
  if(new_permanent_alloc != NULL) /* Parte 1: Pegar 6 Funções + tamanho janela*/
    permanent_alloc = new_permanent_alloc;
  if(new_temporary_alloc != NULL)
    temporary_alloc = new_temporary_alloc;
  permanent_free = new_permanent_free;
  temporary_free = new_temporary_free;
  before_loading_interface = new_before_loading_interface;
  after_loading_interface = new_after_loading_interface;
  window_width = window_width_p;
  window_height = window_height_p;
  {
    int count = -1, i; /* Parte 2: Contar quantas mais funções existem */
    va_list args;
    char *ext;
    va_start(args, new_after_loading_interface);
    do{
      count ++;
      ext = va_arg(args, char *);
      va_arg(args, pointer_to_interface_function*);
    } while(ext != NULL);
    number_of_file_functions_in_the_list = count;
    list_of_file_functions = (struct file_function *)
                               permanent_alloc(sizeof(struct file_function) * 
                                               count); // Parte 3: Alocar o resto
    va_start(args, new_after_loading_interface);
    for(i = 0; i < count; i ++){
      list_of_file_functions[i].extension = va_arg(args, char *);
      list_of_file_functions[i].load_texture =
                          va_arg(args, pointer_to_interface_function*);
    }
  }
  @<Inicialização da API de Interface@>
}
@
\fimcodigo

A função acima ficou verbosa graças ao fato de termos um número
variável de argumentos contendo ponteiros para funções com mais
argumentos. Infelizmente C se torna uma linguagem verbosa nestes
casos. O que infelizmente encobre o quão simples foi o que fizemos na
inicialização acima.

No fim da função deixamos em vermelho espaço para que operações
adicionais que ainda serão definidas sejam feitas.

O uso de alguns recursos como o acesso aos argumentos por meio da
\monoespaco{va\_list} requer a inclusão do cabeçalho abaixo:

\iniciocodigo
@<Cabeçalhos Locais (interface.c)@>+=
#include <stdarg.h>
@
\fimcodigo

\subsecao{3.2. Finalização}

A função complementar à nossa inicialização é a função de
finalização. Se na inicialização terminamos alocando memória para
armazenar as funções que interpretam arquivos, na finaliação
terminaremos desalocando essa mesma memória. Se na inicialização
começamos ajustanto as seis funções personalizadas básicas, na
finalização terminaremos retornando tais funções para seus valores
padrão.

\iniciocodigo
@<Definição de Funções da API (interface.c)@>+=
void _Wfinish_interface(void){
  @<Finalização da API de Interface@>
  if(permanent_free != NULL)
    permanent_free(list_of_file_functions);
  number_of_file_functions_in_the_list = 0;
  permanent_alloc = malloc;
  temporary_alloc = malloc;
  permanent_free = free;
  temporary_free = free;
  before_loading_interface = NULL;
  after_loading_interface = NULL;
}
@
\fimcodigo

Note que deixamos um espaço acima em vermelho para código adicional
que precisarmos colocar na finalização à medida que a API for ficando
mais complexa ao definirmos as outras funções nas seções seguintes.

E isso desfaz tudo o que foi feito na inicialização. É possível
inicializar e finalizar a API várias vezes sem que isso cause qualquer
problema.

\secao{4. Shaders}

Uma das coisas que precisaremos definir é o shader padrão a ser usado
caso o usuário não forneça nenhum personalizado. Isso nos dá uma ótima
oportunidade para apresentar em mais detalhes os requisitos para os
shaders suportados por esta API e como esperaremos que um shader
estejas organizado para que ele seja suportado.

A primeira informação sobre o formato de um shader é qual a linguagem
usada para defini-lo. Usaremos a linguagem GLSL, mas existem várias
versões diferentes para ela. É a primeira linha de todo código GLSL
que define a versão da linguagem. Mas não iremos exigir que o usuário
preencha esta informação, nós faremos isso automaticamente.

Para escolher a versão da linguagem GLSL usada em todos os shaders,
iremos consultar se existe uma macro definida com o nome
\monoespaco{W\_GLSL\_VERSION}. Essa macro será usada para preencher a
primeira linha de todo arquivo GLSL. Caso ela não tenha sido definida
e fornecida pelo usuário, então iremos defini-la vom o valor padrão
``\monoespaco{\#version 100\\n}''. Este valor padrão significa que o
shader usará a linguagem de shader OpenGL ES v1.00. Faremos esse
ajuste de macros bem no início de nosso arquivo com a definição das
funções:

\iniciocodigo
@<Macros Locais (interface.c)@>=
#if !defined(W_GLSL_VERSION)
#define W_GLSL_VERSION "#version 100\n"
#endif
@
\fimcodigo

Há ao menos dois tipos de shaders que precisaremos definir para cada
uma das interfaces diferentes. O primeiro é o shader de vértice, que
irá ser processado para cada vértice de nossa interface. O outro é o
shader de fragmento que processará cada píxel individual. Mas como
indicamos na Inrodução, quando passamos um shader para uma interface,
passamos um único arquivo para ela ao invés de dois. Como poderemos
representar dois shaders diferentes por meio de um só arquivo?

Isso é graças ao fato de que a linguagem de shader GLSL suporta macros
condicionais de pré-processamento, assim como em C. Desta forma,
podemos definir macros diferentes se estamos compilando um shader de
vértice ou se estamos compilando um shader de fragmento. Isso será
feito definindo em cada um dos casos as seguintes macros:

\iniciocodigo
@<Variáveis Locais (interface.c)@>+=
static const char vertex_shader_macro[] = "#define VERTEX_SHADER\n";
static const char fragment_shader_macro[] = "#define FRAGMENT_SHADER\n";
@
\fimcodigo

No código-fonte do shader, poderemos então verificar qual das duas
macros acima estará definida. Dependendo do caso, simplesmente
compilaremos código diferente, da mesma forma como podemos compilar
código diferente em um mesmo arquivo para Linux ou Windows em um mesmo
arquivo com código C.

Em seguida, precisamos definir a precisão padrão para cada tipo de
dados do shader GLSL caso isso não seja declarado explicitamente pelo
usuário ao declarar uma variável. Isso é feito usando a palavra-chave
\monoespaco{precision}, seguida por um qualificador de precisão
(\monoespaco{lowp}, \monoespaco{mediump}, \monoespaco{highp}) e por um
tipo e variável. Na dúvida vamos deixar tudo na precisão mais
alta. Nas variáveis menos importantes, podemos modificar a precisão
para valores menores para obter mais performance.

\iniciocodigo
@<Variáveis Locais (interface.c)@>+=
static const char precision_qualifier[] = "precision highp float;\n"
                                          "precision highp int;\n";
@
\fimcodigo

A próxima coisa que será de nosso interesse será inserir bibliotecas
GLSL. O usuário poderá definir funções adicionais para facilitar o
código GLSL de seus shaders. As bibliotecas ficarão armazenadas na
seguinte variável que iniciará como sendo uma string vazia:

\iniciocodigo
@<Variáveis Locais (interface.c)@>+=
static char *shader_library = "";
@
\fimcodigo

A variável será modificada por meio da função que vimos na Introdução
para incrementar o GLSL com novas funções. Tal função é extremamente
simples e somente irá realizar uma atribuição:

\iniciocodigo
@<Definição de Funções da API (interface.c)@>+=
void _Wset_interface_shader_library(char *source){
  shader_library = source;
}
@
\fimcodigo

Mas não poderemos nos esquecer que na função de finalização
precisaremos desfazer esta mudança. Do contrário, a API pode ser
finalizada e inicializada novamente de modo que a segunda
inicialização fique com as mesmas definições de função feitas na
primeira:

\iniciocodigo
@<Finalização da API de Interface@>=
shader_library = "";
@
\fimcodigo

A lista de variáveis e atributos à disposição de nosso shader será
definida mais adiante. Mas ela também precisa fazer parte do
código-fonte do shader:

\iniciocodigo
@<Variáveis Locais (interface.c)@>+=
static const char shader_variables[] = ""
@<Atributos, Uniformes e Variantes de Shader@>
"";
@
\fimcodigo


Agora vejamos como compilar um shader completo por meio
disso. Primeiro vamos inserir o cabeçalho de entrada e saída padrão
para podermos imprimir mensagens de erro na tela. Os cabeçalhos OpenGL
já foram inseridos em \monoespaco{interface.h}.

\iniciocodigo
@<Cabeçalhos Locais (interface.c)@>+=
#include <stdio.h>
@
\fimcodigo

Uma vez que tenhamos o cabeçalho adequado, vamos definir a função, que
dado um código-fonte de um shader OpenGL, a compila para um programa
de shader completo. Compilar um shader significa criar no OpenGL os
dois tipos de shaders (vértice e fragmento), compilar ambos, e ligá-os
em um único programa.

\iniciocodigo
@<Funções Auxiliares Locais (interface.c)@>+=
@<Funções para Checar Erros de Compilação@>
static GLuint compile_shader(const char *source_code){
  GLuint vertex_shader, fragment_shader, program;
  const char *list_of_source_code[6];
  vertex_shader = glCreateShader(GL_VERTEX_SHADER);
  fragment_shader = glCreateShader(GL_FRAGMENT_SHADER);
  list_of_source_code[0] = W_GLSL_VERSION;
  list_of_source_code[1] = vertex_shader_macro;
  list_of_source_code[2] = precision_qualifier;
  list_of_source_code[3] = shader_library;
  list_of_source_code[4] = shader_variables;
  list_of_source_code[5] = source_code;
  glShaderSource(vertex_shader, 6, list_of_source_code, NULL);
  list_of_source_code[1] = fragment_shader_macro;
  glShaderSource(fragment_shader, 6, list_of_source_code, NULL);
  glCompileShader(vertex_shader);
  if(check_compiling_error(vertex_shader))
    return 0;
  glCompileShader(fragment_shader);
  if(check_compiling_error(fragment_shader))
    return 0;
  program = glCreateProgram();
  glAttachShader(program, vertex_shader);
  glAttachShader(program, fragment_shader);
  glLinkProgram(program);
  if(check_linking_error(program))
    return 0;
  glDeleteShader(vertex_shader);
  glDeleteShader(fragment_shader);
  return program;
}
@
\fimcodigo

A parte que não é mostrada acima é como fazemos a verificação de que
um shader foi compilado com sucesso. Para isso usamos a função
abaixo. Ela funciona para os dois tipos de shader. E consiste em
consultar se ocorreu um erro de compilação. Em caso afirmativo, ela lê
dos logs do OpenGL o que aconteceu de errado e imprime. Note que
usamos as funções temporárias de alocação e desalocação para termos
espaço para a mensagem de erro. E a função retorna se achou ou não um
erro.

\iniciocodigo
@<Funções para Checar Erros de Compilação@>=
static bool check_compiling_error(GLuint shader){
  GLint status;
  glGetShaderiv(shader, GL_COMPILE_STATUS, &status);
  if(status == GL_FALSE){
    int info_log_length;
    char *error_msg;
    glGetShaderiv(shader, GL_INFO_LOG_LENGTH, &info_log_length);
    error_msg = (char *) temporary_alloc(info_log_length);
    glGetShaderInfoLog(shader, info_log_length, &info_log_length, error_msg);
    fprintf(stderr, "Shader Error: %s\n", error_msg);
    if(temporary_free != NULL)
      temporary_free(error_msg);
    return true;
  }
  return false;
}
@
\fimcodigo

Checar se houve um erro quando ligamos os dois shaders funciona de
maneira semehante. Entretanto, podemos fazer ainda melhor. Podemos
tentar validar o shader simulando o seu uso e assim detectar erros
adicionais. Contudo, como isso é uma operação demorada e cara, faremos
isso só se a macro \monoespaco{W\_DEBUG\_INTERFACE} esteja
definida. Neste caso, assumiremos estar em modo de depuração:

\iniciocodigo
@<Funções para Checar Erros de Compilação@>+=
static bool check_linking_error(GLuint program){
  GLint status;
  GLsizei info_log_length;
  char *error_msg;
  glGetProgramiv(program, GL_LINK_STATUS, &status);
  if(status == GL_FALSE){
    glGetProgramiv(program, GL_INFO_LOG_LENGTH, &info_log_length);
    error_msg = (char *) temporary_alloc(info_log_length);
    glGetProgramInfoLog(program, info_log_length, &info_log_length, error_msg);
    fprintf(stderr, "Shader Error: %s\n", error_msg);
    if(temporary_free != NULL)
      temporary_free(error_msg);
    return true;
  }
#if defined(W_DEBUG_INTERFACE)
  glValidateProgram(program);
  glGetProgramiv(program, GL_VALIDATE_STATUS, &status);
  if(status == GL_FALSE){
    glGetProgramiv(program, GL_INFO_LOG_LENGTH, &info_log_length);
    error_msg = (char *) temporary_alloc(info_log_length);
    glGetProgramInfoLog(program, info_log_length, &info_log_length, error_msg);
    fprintf(stderr, "Shader Error: %s\n", error_msg);
    if(temporary_free != NULL)
      temporary_free(error_msg);
    return true;
  }
#endif
  return false;  
}
@
\fimcodigo

E isso termina a descrição de como compilamos um novo shader e
imprimimos erros caso hajam problemas no código-fonte.

Podemos agora apresentar o código do shader padrão a ser usado caso o
usuário não passe um shader personalizado. O objetivo do shader padrão
deve ser apenas mostrar a textura associada à interface caso exista
uma textura.

O código-fonte do shader padrão será armazendado em uma constante:

\iniciocodigo
@<Variáveis Locais (interface.c)@>+=
static const char default_shader_source[] = ""
"#if defined(VERTEX_SHADER)\n"
@<Shader de Vértice Padrão@>
"#else\n"
@<Shader de Fragmento Padrão@>
"#endif\n"
                                           "";
@
\fimcodigo

O código-fonte serão literais strings aramazenadas na variável
constante acima. No caso do shader de vértice, tudo o que faremos por
padrão com cada vértice será multiplicar sua posição pela matriz de
modelo e visualização que armazenará coisas como seu tamanho, posição
e rotação. Além disso, especificamos a posição em nossa textura
associada a cada vértice.

Para entender o código, considere que iremos representar toda
interface usando sempre a mesma sequência de vértices: $(0, 0, 0), (1,
0, 0), (1, 1, 0), (0, 1, 0)$. Cada vértice também possui como atributo
suas coordenadas de textura. Eles são declarados abaixo:

\iniciocodigo
@<Variáveis Locais (interface.c)@>+=
static const float interface_vertices[20] = {
  0.0, 0.0, 0.0, // Primeiro vértice
  0.0, 0.0,     // Coordenada de textura
  1.0, 0.0, 0.0, // Segundo vértice
  1.0, 0.0,     // Textura
  1.0, 1.0, 0.0,  // Terceiro vértice
  1.0, 1.0,      // Textura
  0.0, 1.0, 0.0, // Quarto vértice
  0.0, 1.0};    // Textura
static GLuint interface_vbo;
@
\fimcodigo

A ordem em que declaramos eles é importante. Se seguirmos os vértices
na ordem em que são declarados, percorremos um quadrado no sentido
anti-horário. Essa é a forma de indicarmos que estamos vendo a parte
frontal da face de nossa interface, não a parte traseira. Devemos
estar preparados para caso o usuário confiure o OpenGL para não
mostrar a parte traseira dos vértices, uma otimização bastante comum e
necessária em muitos contextos. Especificando esta ordem garante que
nossa interface estará visível mesmo neste caso.

É importante armazenarmos as texturas rotacionadas para o caso de
termos a macro \monoespaco{W\_FORCE\_LANDSCAPE} definida e termos a
janela com maior altura que largura. Em tais casos nós rotacionamos
nossas interfaces trocando suas larguras por altura e usando a textura
rotacionada.

Estes vértices serão carregados na placa de vídeo a à partir daí
poderão ser acessados por meio do VBO, ou ``vertex buffer object''
correspondente. Carregar os vértices para a placa de vídeo é feito na
inicialização:

\iniciocodigo
@<Inicialização da API de Interface@>+=
glGenBuffers(1, &interface_vbo);
glBindBuffer(GL_ARRAY_BUFFER, interface_vbo);
// Enviando os vértices para a placa de vídeo:
glBufferData(GL_ARRAY_BUFFER, sizeof(interface_vertices), interface_vertices,
             GL_STATIC_DRAW);
@
\fimcodigo

Na finalização devemos destruir o VBO que criamos:

\iniciocodigo
@<Finalização da API de Interface@>+=
glDeleteBuffers(1, &interface_vbo);
@
\fimcodigo

De qualquer forma, isso significa que a interface originalmente será
sempre um retângulo que irá cobrir a tela inteira, pois no OpenGL o
valor de 1 vai ser sempre tanto a altura como largura da tela.  Para
transformar esse retângulo fixo em coisas de diferentes tamanhos,
posições e rotações nós usamos diferentes matrizes de modelo e
visualização e multiplicamos a posição fixa pela matriz que será
diferente para cada interface.

\iniciocodigo
@<Shader de Vértice Padrão@>=
"void main(){\n"
"  gl_Position = model_view_matrix * vec4(vertex_position, 1.0);\n"
"  texture_coordinate = vertex_texture_coordinate;\n"
"}\n"
@
\fimcodigo

No shader de fragmento, o que faremos para cada pixel é desenhar o
pixel da textura associado à esta posição.

\iniciocodigo
@<Shader de Fragmento Padrão@>=
"void main(){\n"
"  vec4 texture = texture2D(texture1, texture_coordinate);\n"
"  gl_FragData[0] = texture;\n"
"}\n"
@
\fimcodigo

Agora temos que definir para os shaders seus atributos, uniformes e
variantes. Os atributos são variáveis somente-leitura especificados
para cada vértice. Os únicos atributos que teremos é a posição do
vértice, em coordenada $(x, y, z)$ e a coordenada da textura na forma
$(x, y)$. Nos dois casos as coordenadas $x$ e $y$ irão variar entre 0
e 1 e assumiremos $z=0$ como descrito antes. Atributos só podem ser
declarados no shader de vértice.

\iniciocodigo
@<Atributos, Uniformes e Variantes de Shader@>=
"#if defined(VERTEX_SHADER)\n"
"attribute vec3 vertex_position;\n"
"attribute vec2 vertex_texture_coordinate;\n"
"#endif\n"
@
\fimcodigo

Uniformes são variáveis que também são passadas para vértices, mas
eles não irão nunca mudar entre um vértice e outro e nunca precisarão
também ser interpolados. Iremos manter como uniformes a cor de frente
e de fundo, ma matriz de modelo e visualização, o tamanho do objeto em
pixels, o tempo atual em segundos, o inteiro associado a cada
interface e também sua textura:

\iniciocodigo
@<Atributos, Uniformes e Variantes de Shader@>+=
"uniform vec4 foreground_color, background_color;\n"
"uniform mat4 model_view_matrix;\n"
"uniform vec2 interface_size;\n"
"uniform vec2 mouse_coordinate;\n"
"uniform float time;\n"
"uniform int integer;\n"
"uniform sampler2D texture1;\n"
@
\fimcodigo

Por fim, temos as variantes. Estas variáveis podem ser modificadas e
preenchidas pelo shader de vértice. Seu valor interpolado será passado
para o shader de fragmento. É aqui que declaramos a coordenada em
textura que iremos usar.

\iniciocodigo
@<Atributos, Uniformes e Variantes de Shader@>+=
"varying mediump vec2 texture_coordinate;\n"
@
\fimcodigo

Note que nós não usamos todas as variáveis definidas no shader
padrão. Durante a compilação muitas delas serão simplesmente ignoradas
e descartadas como otimização. Mas declaramos mesmo as que não usamos
apenas para informar quais são as variáveis que shaders personalizados
definidos pelo usuário pode usar.

Falando em shaders personalizados pelo usuário, eles serão fornecidos
pela função \monoespaco{\_Wnew\_interface} na forma de uma string com
o nome do arquivo em que ela estará. Para tal caso, vamos também
fornecer uma função que cria um novo shader à partir do conteúdo de um
arquivo cujo caminho é passado como argumento:

\iniciocodigo
@<Funções Auxiliares Locais (interface.c)@>=
static GLuint compile_shader_from_file(const char *filename){
  char *buffer;
  size_t source_size, ret;
  FILE *fp;
  GLuint shader_program;
  fp = fopen(filename, "r");
  if(fp == NULL)  return 0;
  // Vai pro fim do arquivo para ler o tamanho e volta pro começo:
  fseek(fp, 0, SEEK_END);
  source_size = ftell(fp);
  // Aloca e lê buffer
  buffer = (char *) temporary_alloc(sizeof(char) * (source_size + 1));
  if(buffer == NULL) return 0;
  do{
    rewind(fp);
    ret = fread(buffer, sizeof(char), source_size, fp);
  } while(feof(fp) && !ferror(fp) && ret / sizeof(char) == source_size);
  buffer[source_size] = '\0';
  shader_program = compile_shader(buffer);
  if(temporary_free != NULL) temporary_free(buffer);
  return shader_program;
}
@
\fimcodigo


Uma última preocupação que teremos aqui é o que renderizar como textura
padrão quando um shader não tem textura alguma. Para isso, vamos criar
uma textura de apenas 1 pixel branco:

\iniciocodigo
@<Variáveis Locais (interface.c)@>+=
static GLuint default_texture;
@
\fimcodigo

Criamos tal textura branca na inicialização. Isso é feito gerando a
textura no OpenGL, a associando à textura bidimensional atual e
especificando seu único pixel branco.

\iniciocodigo
@<Inicialização da API de Interface@>+=
{
  GLubyte pixels[3] = {255, 255, 255};
  glGenTextures(1, &default_texture);
  glBindTexture(GL_TEXTURE_2D, default_texture);
  glTexImage2D(GL_TEXTURE_2D, 0, GL_RGB, 1, 1, 0, GL_RGB, GL_UNSIGNED_BYTE,
               pixels);
}
@
\fimcodigo

E na finalização descartamos a textura:

\iniciocodigo
@<Finalização da API de Interface@>+=
glDeleteTextures(1, &default_texture);
@
\fimcodigo

\secao{5. A Matriz de Modelo-Visualização}

Como vimos, toda interface será representada por quatro vértices fixos
de tamanho 1. O que modificará na prática o tamanho, posição e rotação
de nossos vértices será a matriz de modelo-visualização que usaremos
para cada interface.

Para entender a matriz, primeiro lembre-se que quando lidamos com ela
no nosso shader de vértice padrão, cujo código mostramos
anteriormente, nós convertemos cada vértice como um vértice em 4
dimensões ao invés de duas ou três. O trecho de código
\monoespaco{vec4{vertex\_position, 1.0)}} basicamente pega uma
coordenada em três dimensões e adiciona uma quarta dimensão sempre
igual a 1.

Nós fazemos isso porque somente em quatro dimensões é possível
representar todas as operações necessárias como a rotação, translação
e escala de nossa interface somente por meio de multiplicação de
matrizes. Ou seja, somente se usarmos quatro dimensões tais operações
são lineares. Placas de vídeo são bastante rápidas quando estão
fazendo esse tipo de operação, enquanto fazer outras operações gera
impacto maior na performance.

Antes de chegar ao formato final da matriz modelo-visualização, vamos
primeiro observar as suas diferentes partes. Primeiro, vamos assumir
que temos um vetor de 4 dimensões que multiplicará nossa matriz. Se
estamos apenas interessados na translação de seus pontos, movendo eles
sem modificar o tamanho, podemos multiplicá-lo por uma matriz com a
forma abaixo:

$$ \left[{{1 \atop 0}\atop {0\atop 0}}{{0 \atop 1}\atop {0 \atop
      0}}{{0\atop 0}\atop{1 \atop 0}}{{x \atop y}\atop{0\atop
      1}}\right]\left[{{x_0\atop y_0}\atop {z_0\atop
      1}}\right]=\left[{{1x_0+0y_0+0z_0+x\atop 0x_0+1y_0+0z_0+y}\atop
    {0x_0+0y_0+1z_0+0\atop
      0x_0+0y_0+0z_0+1}}\right]=\left[{{x_0+x\atop y_0+y}\atop
    {z_0\atop 1}}\right]
$$

Já para mudar o tamanho de uma interface, tanto a largura como a
altura, assumindo que ela está centralizada na origem $(0, 0, 0, 1)$,
podemos multiplicar o vetor pela seguinte matriz de escala::

$$ \left[{{w \atop 0}\atop {0\atop 0}}{{0 \atop h}\atop {0 \atop
      0}}{{0\atop 0}\atop{1 \atop 0}}{{0 \atop 0}\atop{0\atop
      1}}\right]\left[{{x_0\atop y_0}\atop {z_0\atop
      1}}\right]=\left[{{wx_0+0y_0+0z_0+0\atop 0x_0+hy_0+0z_0+0}\atop
    {0x_0+0y_0+1z_0+0\atop 0x_0+0y_0+0z_0+1}}\right]=\left[{{wx_0\atop
      hy_0}\atop {z_0\atop 1}}\right]
$$

E finalmente, no caso da rotação, para rotacionar $\theta$ radianos,
nós podemos usar a seguinte matriz que produzirá o resultado correto
de acordo com a fórmula trigonométrica da rotação:

$$ \left[{{cos(\theta) \atop \sin(\theta)}\atop {0\atop
      0}}{{-sin(\theta) \atop cos(\theta)}\atop {0 \atop 0}}{{0\atop
      0}\atop{1 \atop 0}}{{0 \atop 0}\atop{0\atop
      1}}\right]\left[{{x_0\atop y_0}\atop {z_0\atop
      1}}\right]=\left[{{x_0 cos(\theta)-y_0 sin(\theta)+0z_0+0\atop
      x_0 sin(\theta)+y_0 cos(\theta)+0z_0+0}\atop
    {0x_0+0y_0+1z_0+0\atop 0x_0+0y_0+0z_0+1}}\right]=\left[{{x_0
      cos(theta) - y_0 sin(\theta) \atop x_0 sin(\theta) + y_0
      cos(\theta)}\atop {z_0\atop 1}}\right]
$$

Para formar a matriz de modelo-visualização nós iremos multiplicar
matrizes como estas três, dependendo dos valores exator que precisamos
ajustar o tamanho, posição e rotação de cada interface. Contudo, a
ordem em que a multiplicação de matrizes ocorre faz diferença no
resultado. A ordem correta na qual devemos fazer as operações é:

1. Devemos centralizar o quadrado na origem do OpenGL. (Matriz A)

2. Aumentamos ou diminuimos o tamanho da interface de acordo com seu
tamanho em pixels. (Matriz B)

3. Rotacionamos ela o ângulo necessário. (Matriz C)

4. Redimencionamos ela novamente, desta vez para estar de acordo com
as proporções da nossa área de desenho. Isso muda as coordenadas de
pixels para coordenadas OpenGL. (Matriz D)

5. Fazemos a translação dela para a posição desejada de acordo com a
coordenada OpenGL. (Matriz E)

$$
E(D(C(B(A v)))) = ((((ED)C)B)A)v
$$

Se usarmos outra ordem, coisas podem dar errado. Por exemplo, ao invés
de rotacionar a interface usando o seu centro como eixo, usaríamos o
seu canto inferior direito ou alguma outra posição como eixo.

Ao invés de toda vez termos que multiplicar toda vez essas quatro
matrizes, podemos calcular agora qual vai ser o formato da matriz que
resulta de toda essa multiplicação. Essa matriz será a matriz de
modelo-visualização final. Por exemplo, se $w_w$ é a largura da tela e
$h_w$ é a altura, multiplicar as matrizes $E$ e $D$ resulta em:

$$ ED=\left[{{1 \atop 0}\atop {0\atop 0}}{{0 \atop 1}\atop {0 \atop
      0}}{{0\atop 0}\atop{1 \atop 0}}{{x \atop y}\atop{0\atop
      1}}\right]\left[{{2/w_w \atop 0}\atop {0\atop 0}}{{0 \atop
      2/h_w}\atop {0 \atop 0}}{{0\atop 0}\atop{1 \atop 0}}{{0 \atop
      0}\atop{0\atop 1}}\right]=
\left[{{2/w_w \atop 0}\atop {0\atop 0}}
  {{0 \atop 2/h_w}\atop {0 \atop 0}}
  {{0\atop 0}\atop{1 \atop 0}}
  {{x \atop y}\atop{0\atop 1}}\right]
$$

Note que estamos usando $2/w_w$ e $2/h_w$ porque assumimos que a
altura e largura de uma janela em OpenGL é sempre 2. Dessa forma, um
objeto com largura igual a $w_w$ deve passar a ter largura 2 para
ocupar a tela inteira. O mesmo ocorre com a altura.

Se multiplicarmos pela matriz C, obtemos:

$$
EDC = \left[{{2/w_w \atop 0}\atop {0\atop 0}} {{0 \atop 2/h_w}\atop
  {0 \atop 0}} {{0\atop 0}\atop{1 \atop 0}} {{x \atop y}\atop{0\atop
  1}}\right]\left[{{cos(\theta) \atop \sin(\theta)}\atop {0\atop
  0}}{{-sin(\theta) \atop cos(\theta)}\atop {0 \atop 0}}{{0\atop
  0}\atop{1 \atop 0}}{{0 \atop 0}\atop{0\atop 1}}\right] =
\left[{{2/w_w\cdot cos(\theta) \atop 2/h_w\cdot sin(\theta)}
\atop {0\atop 0}}
  {{-2/w_w\cdot sin(\theta) \atop 2/h_w\cdot cos(\theta)}\atop
  {0 \atop 0}} {{0\atop 0}\atop{1 \atop 0}} {{x \atop y}\atop{0\atop
  1}}\right]
$$

Agora se multiplicarmos este resultado pela matriz $B$, levando em
conta que $h_p$ é a altura em pixels e $w_p$ é a largura em pixels,
obtemos:

$$ EDCB =
\left[{{2/w_w\cdot cos(\theta) \atop 2/h_w\cdot sin(\theta)}
\atop {0\atop 0}}
  {{-2/w_w\cdot sin(\theta) \atop 2/h_w\cdot cos(\theta)}\atop
  {0 \atop 0}} {{0\atop 0}\atop{1 \atop 0}} {{x \atop y}\atop{0\atop
  1}}\right]\left[{{w_p \atop 0}\atop {0\atop 0}}{{0 \atop h_p}\atop
  {0 \atop 0}}{{0\atop 0}\atop{1 \atop 0}}{{0 \atop 0}\atop{0\atop
  1}}\right]=
\left[{{2w_p/w_w\cdot cos(\theta) \atop 2w_p/h_w\cdot sin(\theta)}
\atop {0\atop 0}}
  {{-2h_p/w_w\cdot sin(\theta) \atop 2h_p/h_w\cdot cos(\theta)}\atop
  {0 \atop 0}} {{0\atop 0}\atop{1 \atop 0}} {{x \atop y}\atop{0\atop
  1}}\right]
$$

E finalmente, multiplicamos este resultado pela matriz $A$, que
centraliza em $(0, 0, 0, 1)$ toda interface. Note que originalmente
a interface não-multiplicada pela matriz de modelo-visuzlização está
centralizada em $(1/2, 1/2, 0, 1)$. Então a matriz $A$ é simplesmente
uma matriz de translação com valores constantes deslocando 1/2 para a
esquerda e 1/2 para baixo:

$$
EDCBA =\left[{{2w_p/w_w\cdot cos(\theta) \atop 2w_p/h_w\cdot sin(\theta)}
\atop {0\atop 0}}
  {{-2h_p/w_w\cdot sin(\theta) \atop 2h_p/h_w\cdot cos(\theta)}\atop
  {0 \atop 0}} {{0\atop 0}\atop{1 \atop 0}} {{x \atop y}\atop{0\atop
  1}}\right]
\left[{{1 \atop 0}\atop {0\atop 0}}{{0 \atop 1}\atop {0 \atop
      0}}{{0\atop 0}\atop{1 \atop 0}}{{-1/2 \atop -1/2}\atop{0\atop
      1}}\right]=
$$

$$
=\left[{{2w_p/w_w\cdot cos(\theta) \atop 2w_p/h_w\cdot sin(\theta)}
\atop {0\atop 0}}
  {{-2h_p/w_w\cdot sin(\theta) \atop 2h_p/h_w\cdot cos(\theta)}\atop
  {0 \atop 0}} {{0\atop 0}\atop{1 \atop 0}} {{x+h_p/w_w\cdot
  sin(\theta)-w_p/w_w\cdot cos(\theta) \atop y-w_p/h_w\cdot
  sin(\theta)-h_p/h_w\cdot cos(\theta)}\atop{0\atop 1}}\right]
$$

O formato acima então é a forma final da matriz de modelo-visualização
de toda interface que iremos armazenar.

Entretanto, os valores $(x, y)$ acima são valores nas coordenadas
OpenGL. Mas nós medimos as coordenadas de forma difeente. Para nós a
origem não é o centro da tela, mas o canto esquerdo inferior da
tela. Para nós a posição na tela é medida também em píxels, e não
sempre o mesmo tamanho constante de 2 como para o OpenGL. Então a
função que irá preencher esta matriz precisa também realizar a
conversão.

No caso, se temos um ponteiro para uma interface
chamado \monoespaco{i}, a conversão segue as seguintes regras:

\iniciocodigo
@<Conversão de Coordenadas e Tamanho@>=
x = 2.0 * (i -> _x) / (*window_width) - 1.0;
y = 2.0 * (i -> _y) / (*window_height) - 1.0;
@
\fimcodigo

Usando a conversão acima, a função abaixo preenche a matriz de
modelo-visualização pela primeira vez:

\iniciocodigo
@<Funções Auxiliares Locais (interface.c)@>=
static void initialize_model_view_matrix(struct user_interface *i){
  GLfloat x, y;
  @<Conversão de Coordenadas e Tamanho@>
  GLfloat cos_theta = cos(i -> _rotation);
  GLfloat sin_theta = sin(i -> _rotation);
  /* Primeira Coluna */
  i -> _transform_matrix[0] = (2 * i -> width / (*window_width)) *
    cos_theta;
  i -> _transform_matrix[1] = (2 * i -> width / (*window_height)) *
    sin_theta;
  i -> _transform_matrix[2] = 0.0;
  i -> _transform_matrix[3] = 0.0;
  /* Segunda Coluna */
  i -> _transform_matrix[4] = -(2 * i -> height / (*window_width)) *
     sin_theta;
  i -> _transform_matrix[5] = (2 * i -> height / (*window_height)) *
    cos_theta;
  i -> _transform_matrix[6] = 0.0;
  i -> _transform_matrix[7] = 0.0;
  /* Terceira Coluna */
  i -> _transform_matrix[8] = 0.0;
  i -> _transform_matrix[9] = 0.0;
  i -> _transform_matrix[10] = 1.0;
  i -> _transform_matrix[11] = 0.0;
  /* Quarta Coluna */
  i -> _transform_matrix[12] = x +
    (i -> height / (*window_width)) * sin_theta -
    (i -> width / (*window_width)) * cos_theta;
  i -> _transform_matrix[13] = y -
    (i -> width / (*window_height)) * sin_theta -
    (i -> height / (*window_height)) * cos_theta;
  i -> _transform_matrix[14] = 0.0;
  i -> _transform_matrix[15] = 1.0;
}
@
\fimcodigo

O uso das funções de seno e cosseno requer que coloquemos o cabeçalho
de funções matemáticas:

\iniciocodigo
@<Cabeçalhos Locais (interface.c)@>+=
#include <math.h>
@
\fimcodigo

\secao{6. Gerenciando as Estruturas de Dados}

Vimos que as estruturas de dados que nossa API irá gerenciar é uma
estrutura para interfaces, outra para ligação para interface
existente, uma terceira para fazer marcação no histórico e por fim,
uma lista encadeada que manterá a sequência e o histórico de criação
para os três tipos de interface anteriores.

A nossa lista encadeada pode ser acessada e percorrida através de dois
ponteiros. O primeiro apontará para a última estrutura criada e o
segundo apontará para a última marcação no hstórico criada. Também
iremos usar um mutex para garantir que somente uma thread possa
modificar a lista encadeada:

\iniciocodigo
@<Variáveis Locais (interface.c)@>+=
static void *last_structure = NULL;
static struct marking *last_marking = NULL;
_STATIC_MUTEX_DECLARATION(linked_list_mutex);
@
\fimcodigo

O mutex que controla acesso à lista encadeada precisa ser criado na
inicialização:

\iniciocodigo
@<Inicialização da API de Interface@>+=
MUTEX_INIT(&linked_list_mutex);
@
\fimcodigo

Deixaremos para definir a finalização deste mutex junto com a
finalização de toda a lista encadeada nas próximas seções.

\subsecao{6.1. Criando e Destruindo Estruturas de Shader}

Na Seção 4 nós mostramos o código-fonte do shader padrão e também
mostramos funções que fazem a compilação do código-fonte de um
shader. Mas nós não mostramos como aquelas funções são intergradas na
função que constrói a estrutura de shader. Faremos isso nesta seção.

A nossa API não tem nenhuma função definida para gerar um novo shader
por si só. Ao invés disso, nós iremos criar um novo shader por meio de
uma função interna toda vez que formos criar uma nova interface e o
usuário passar um shader personalizado. Ou seja, quando o usuário da
API invoca \monoespaco{\_Wnew\_interface} com o segundo parâmetro, que
é o do arquivo com código-fonte de shader, diferente
de \monoespaco{NULL}.

Também precisamos gerar a estrutura para o shader padrão, cujo código
apresentamos na Seção 4.

O shader padrão será armazenado no seguinte ponteiro:

\iniciocodigo
@<Variáveis Locais (interface.c)@>+=
struct shader *default_shader;
@
\fimcodigo

Vamos definir agora uma função interna que irá gerar uma nova
estrutura de shader. Ela recebe como argumento a string com nome de
arquivo onde está o código-fonte do shader. A função retorna o
ponteiro para a estrutura gerada e também a insere na lista
encadeada. Exceto se o nome de arquivo passado
for \monoespaco{NULL}. Neste caso, nós apenas geramos uma estrutura à
partir do código-fonte do shader padrão definido na Seção 4. Ele não é
armazenado na lista encadeada.

\iniciocodigo
@<Funções Auxiliares Locais (interface.c)@>+=
static struct shader *new_shader(char *shader_source){
  struct shader *new = (struct shader *) permanent_alloc(sizeof(struct shader));
  if(new != NULL){
    new -> type = TYPE_SHADER;
    new -> next = NULL;
    if(shader_source == NULL)
      new -> program = compile_shader(default_shader_source);
    else
      new -> program = compile_shader_from_file(shader_source);
    // Estabelecendo atributos
    glBindAttribLocation(new -> program, 0, "vertex_position");
    glBindAttribLocation(new -> program, 1, "vertex_texture_coordinate");
    // Obtendo uniformes:
    new -> uniform_foreground_color =  glGetUniformLocation(new -> program,
                                                            "foreground_color");
    new -> uniform_background_color =  glGetUniformLocation(new -> program,
                                                            "background_color");
    new -> uniform_model_view_matrix = glGetUniformLocation(new -> program,
                                                            "model_view_matrix");
    new -> uniform_interface_size = glGetUniformLocation(new -> program,
                                                         "interface_size");
    new -> uniform_mouse_coordinate = glGetUniformLocation(new -> program,
                                                         "mouse_coordinate");
    new -> uniform_time = glGetUniformLocation(new -> program, "time");
    new -> uniform_integer = glGetUniformLocation(new -> program, "integer");
    new -> uniform_texture1 = glGetUniformLocation(new -> program, "texture1");
    if(shader_source != NULL){ // Insere na lista encadeada:
      MUTEX_WAIT(&linked_list_mutex); // Preparando mutex
      if(last_structure != NULL)
        ((struct user_interface *) last_structure)-> next = (void *) new;
      last_structure = (void *) new;
      MUTEX_SIGNAL(&linked_list_mutex);
    }
  }
  return new;
}
@
\fimcodigo

Basicamente o código acima usa a função da Seção 4 para compilar o
shader, e uma vez que tem o shader compilado, armazena na estrutura de
shader a posição de cada uma das variáveis do shader compilado que
podem ser modificadas antes de executar o shader.

Lembrando que o shader padrão precisa ser compilado e criado na
Inicialização. O código para isso é:

\iniciocodigo
@<Inicialização da API de Interface@>+=
default_shader = new_shader(NULL);
@
\fimcodigo

Para destruir um shader criado, só precisamos informar que não
usaremos mais o programa de shader compilado e que ele deve ser
apagado. Em seguida, desalocamos a estrutura de shader que o
armazenava:

\iniciocodigo
@<Funções Auxiliares Locais (interface.c)@>+=
static void destroy_shader(struct shader *shader_struct){
  glDeleteProgram(shader_struct -> program);
  if(permanent_free != NULL)
    permanent_free(shader_struct);
}
@
\fimcodigo

O shader padrão deve ser destruído na finalização da API:

\iniciocodigo
@<Finalização da API de Interface@>+=
destroy_shader(default_shader);
@
\fimcodigo

\subsecao{6.2. Criando e Destruindo Interface}

Vimos que a criação de novas interfaces se dará pela função
\monoespaco{\_Wnew\_interface}. Esta função deverá alocar a nova
interface, executar as funções adequadas para inicializá-la e
incluí-la na lista encadeada atualizando os ponteiros de maneira
adequada. A função será definida como:

\iniciocodigo
@<Definição de Funções da API (interface.c)@>+=
struct user_interface *_Wnew_interface(char *filename, char *shader_filename,
                                  float x, float y, float z, float width,
                                  float height){
  struct user_interface *new_interface;
  void (*loading_function)(void *(*permanent_alloc)(size_t),
                           void (*permanent_free)(void *),
                           void *(*temporary_alloc)(size_t),
                           void (*temporary_free)(void *),
                           void (*before_loading_interface)(void),
                           void (*after_loading_interface)(void),
                           char *source_filename, struct user_interface *target);
  int i;
  new_interface = permanent_alloc(sizeof(struct user_interface));
  if(new_interface != NULL){
    new_interface -> type = TYPE_INTERFACE;
    new_interface -> next = NULL;
    new_interface-> x = new_interface-> _x = x;
    new_interface -> y = new_interface-> _y = y;
    new_interface -> rotation = new_interface -> _rotation = 0;
#if defined(W_FORCE_LANDSCAPE)
   if(*window_height > *window_width){
      new_interface-> _x = *window_width - y;
      new_interface -> _y = x;
      new_interface -> _rotation += M_PI_2;
   }
#endif
    new_interface -> z = z;
    new_interface -> width = width;
    new_interface -> height = height;
    for(i = 0; i < 4; i ++){
      new_interface -> background_color[i] = 0.0;
      new_interface -> foreground_color[i] = 0.0;
    }
    new_interface -> integer = 0;
    new_interface -> visible = true;
    initialize_model_view_matrix(new_interface);
    if(shader_filename != NULL)
      new_interface -> shader_program = new_shader(shader_filename);
    else
      new_interface -> shader_program = default_shader;
    new_interface -> _texture1 = NULL;
    if(filename != NULL) // Ainda tem que carregar a textura:
      new_interface -> _loaded_texture = false;
    else // Sem textura para ser carregada:
      new_interface -> _loaded_texture = true;
    new_interface -> animate = false;
    new_interface -> number_of_frames = 0;
    new_interface -> current_frame = 0;
    new_interface -> frame_duration = NULL;
    new_interface -> _t = 0;
    new_interface -> max_repetition = 0;
    MUTEX_INIT(&(new_interface -> mutex));
    new_interface -> _mouse_over = false;
    new_interface -> on_mouse_over = NULL;
    new_interface -> on_mouse_out = NULL;
    new_interface -> on_mouse_left_down = NULL;
    new_interface -> on_mouse_left_up = NULL;
    new_interface -> on_mouse_middle_down = NULL;
    new_interface -> on_mouse_middle_up = NULL;
    new_interface -> on_mouse_right_down = NULL;
    new_interface -> on_mouse_right_up = NULL;
    MUTEX_WAIT(&linked_list_mutex); // Inserindo na lista encadeada
    if(last_structure != NULL)
      ((struct user_interface *) last_structure)-> next = (void *) new_interface;
    last_structure = (void *) new_interface;
    last_marking -> number_of_interfaces ++;
    MUTEX_SIGNAL(&linked_list_mutex);
    if(filename != NULL){ // Get and run loading function:
      char *ext;
      for(ext = filename; *ext != '\0'; ext ++);
      for(; *ext != '.' && ext != filename; ext --);
      if(*ext == '.'){
        ext ++;
        loading_function = get_loading_function(ext);
        if(loading_function != NULL)
          loading_function(permanent_alloc, permanent_free, temporary_alloc,
                           temporary_free, before_loading_interface,
                           after_loading_interface, filename, new_interface);
      }
    }
  }
  return new_interface;
}
@
\fimcodigo

Apesar de longa, tudo o que a função acima faz é alocar a interface
com a função de alocação permanente configurada, inicializar os
valores iniciais, inseri-la na lista encadeada e executar a função
adequada para carrega a textura dependendo da extensão do arquivo.

Destruir uma interface significa checar algumas de suas variáveis
específicas para vcer se tem um valor diferente
de \monoespaco{NULL}. Se for o caso, isso significa que a função de
carregamento alocou e preencheu alguns de seus valores. Como por
exemplo, as variáveis que armazenam texturas, que armazenam o tempo
para cada frame de animação se for uma textura animada. A função de
destruição de interfaces desaloca primeiro essas coisas antes de
desalocar a estrutura de interface em si. Ela também deve finalizar o
mutex da interface.

\iniciocodigo
@<Funções Auxiliares Locais (interface.c)@>+=
static void destroy_interface(struct user_interface *interface_struct){
  if(interface_struct -> _texture1 != NULL){
    glDeleteTextures(interface_struct -> number_of_frames,
                     interface_struct -> _texture1);
    if(permanent_free != NULL)
      permanent_free(interface_struct -> _texture1);
  }
  if(interface_struct -> frame_duration != NULL && permanent_free != NULL)
    permanent_free(interface_struct -> frame_duration);
  MUTEX_DESTROY(&(interface_struct -> mutex));
  if(permanent_free != NULL)
    permanent_free(interface_struct);
}
@
\fimcodigo

\subsecao{6.3. Criando e Destruindo Ligação para Interface Existente}

Como vimos, além de criar uma nova interface podemos apenas criar uma
ligação para uma interface existente, que ela irá se comportar como
uma interface recém-criada para fins de interação. Uma ligação será
apenas um ponteiro para outra interface em nossa lista encadeada. E
elas são criadas com a função \monoespaco{\_Wlink\_interface}. A
definição da função é:

\iniciocodigo
@<Definição de Funções da API (interface.c)@>+=
struct user_interface *_Wlink_interface(struct user_interface *i){
  struct link *new_link = permanent_alloc(sizeof(struct link));
  if(new_link == NULL)
    return NULL;
  new_link -> type = TYPE_LINK;
  new_link -> next = NULL;
  new_link -> linked_interface = i;
  MUTEX_WAIT(&linked_list_mutex); // Inserindo na lista encadeada
  if(last_structure != NULL)
    ((struct user_interface *) last_structure)-> next = (void *) new_link;
  last_structure = (void *) new_link;
  last_marking -> number_of_interfaces ++;
  MUTEX_SIGNAL(&linked_list_mutex);
  return i;
}
@
\fimcodigo

Destruir uma interface de ligação significa apenas executar a função
de desalocação e por isso não criaremos uma função auxiliar para isso.


\subsecao{6.4. Criando Marcação no Histórico de Interfaces}

Para criar uma nova marcação no histórico de interfaces, o usuário
deve usar a função \monoespaco{\_Wmark\_history\_interface}. Depois da
marcação, todas as interfaces criadas antes ficarão inacessíveis até
que a marcação seja removida.

\iniciocodigo
@<Definição de Funções da API (interface.c)@>+=
void _Wmark_history_interface(void){
  struct marking *new_marking = permanent_alloc(sizeof(struct marking));
  if(new_marking != NULL){
    new_marking -> type = TYPE_MARKING;
    new_marking -> next = NULL;
    new_marking -> previous_marking = last_marking;
    new_marking -> number_of_interfaces = 0;
    MUTEX_WAIT(&linked_list_mutex); // Inserindo na lista encadeada
    new_marking -> prev = last_structure;
    if(last_structure != NULL)
      ((struct user_interface *) last_structure)-> next = (void *) new_marking;
    last_structure = (void *) new_marking;
    last_marking = new_marking;
    MUTEX_SIGNAL(&linked_list_mutex);
  }
}
@
\fimcodigo

Como é importante existir uma última marcação no hstórico para termos
uma contagem de quantas interfaces estão ativas em um dado momento,
vamos criar a primeira marcação durante a inicialização da API. Esta
primeira marcação nunca será apagada, exceto na finalização da
API. Mesmo que o usuário tente removê-la antes por meio de funções.

\iniciocodigo
@<Inicialização da API de Interface@>+=
_Wmark_history_interface();
@
\fimcodigo

\iniciocodigo
@<Finalização da API de Interface@>+=
// Apaga todas as marcações, menos a primeira:
while(last_marking -> previous_marking != NULL){
  _Wrestore_history_interface();
}
// Apaga as interfaces após a primeira marcação:
_Wrestore_history_interface();
// Apaga a primeira marcação:
if(permanent_free != NULL)
  permanent_free(last_marking);
last_marking = NULL;
last_structure = NULL;
MUTEX_DESTROY(&linked_list_mutex);
@
\fimcodigo

A função \monoespaco{\_Wrestore\_history\_interface} é uma das funções
de API que ainda não definimos. Ela destrói e remove todas as
estruturas criadas após a última marcação no histórico e a última
marcação. Exceto no caso da primeira marcação que nunca é removida. E
por causa disso, no código acima, esta primeira marcação é removida
manualmente com um \monoespaco{permanenet\_free}.

\subsecao{6.5. Removendo Marcações e Interfaces}

A forma de remover e desalocar interfaces é através da restauração do
histórico, removendo a última marcação e voltando ao estado que
estávamos quando ela foi criada. Isso remove toda e qualquer interface
que tenha sido criada após a marcação removida. Isso é feito usando a
função
\monoespaco{\_Wrestore\_history\_interface}.

O que esta função faz é remover a última marcação de histórico em
nossa lista encadeada, fazendo com que a última marcação volte a ser
aquela que existia antes. Se não existir nenhuma outra marcação além
da que existe, nós não a removemos. Mas independente disso, nós de
qualquer forma percorremos a lista encadeada e removemos todas as
interfaces (e ligações) que estiverem depois dela na lista.

A implementação da função é:

\iniciocodigo
@<Definição de Funções da API (interface.c)@>+=
void _Wrestore_history_interface(void){
  struct marking *to_be_removed;
  struct user_interface *current, *next;
  MUTEX_WAIT(&linked_list_mutex);
  last_structure = last_marking -> prev;
  if(last_structure != NULL)
    ((struct user_interface *) last_structure) -> next = NULL;
  to_be_removed = last_marking;
  current = (struct user_interface *) to_be_removed -> next;
  // Removendo estruturas criadas após última marcação:
  while(current != NULL){
    next = (struct user_interface *) (current -> next);
    if(current -> type == TYPE_INTERFACE)
      destroy_interface(current);
    else if(current -> type == TYPE_SHADER)
      destroy_shader((struct shader *) current);
    else if(permanent_free != NULL)
      permanent_free(current);
    current = next;
  }
  // Removendo última marcação se não for a primeira
  if(to_be_removed -> previous_marking != NULL){
    last_marking = to_be_removed -> previous_marking;
    if(permanent_free != NULL)
      permanent_free(to_be_removed);
  }
  else
    to_be_removed -> number_of_interfaces = 0;
  @<Restauração de Histórico@>
  MUTEX_SIGNAL(&linked_list_mutex);
}
@
\fimcodigo

Basicamente a função acima primeiro remove todas as estruturas de
dados criadas após a última marcação. Só depois de fazer isso que a
marcação é então removida. Exceto se não houver nenhuma marcação
antes, o que significa que devemos mantê-la. O ponteiro para o último
elemento da nossa lista encadeada é atualizado logo no começo da
função, pois a última marcação armazena qual era o último elemento
antes dela ser criada por meio de seu ponteiro para o elemento
anterior.

\secao{7. Renderizando Interfaces}

Para renderizar as interfaces, devemos levar em conta a ordem correta
em que cada uma delas deve ser renderizada. Uma das formas mais
simples de se lidar com isso é simplesmente desenhar em qualquer ordem
e confiar na coordenada $z$ para que o OpenGL saiba o que desenhar na
frente por meio de um simples teste de profundidade. Contudo, ao lidar
com objetos trasparentes, isso nem sempre gera o resultado correto. Se
primeiro desenharmos um objeto transparente e depois um objeto opaco
logo atrás dele, partes que deveriam aparecer do objeto atrás por
causa da transparêndia do objeto na frente podem não aparecer.

No caso de nossas interfaces, como estamos sempre lidando com objetos
bidimensionais que sempre estarão paralelos em relação ao eixo $z$, o
que faremos é apenas manter uma lista ordenada de ponteiros para
interfaces. A lista estará ordenada de modo que objetos com valores
menores da coordenada $z$ apareçam antes dos objetos com coordenadas
de valores maiores. Desta forma, objetos que são desenhados na frente
serão desenhados sempre depois de objetos posicionados atrás
deles. Basicamente este é o chamado ``algoritmo do pintor''. Tal como
o pintor de um quadro, começaremos sempre desenhando aquilo que
estiver mais distante.

Para interfaces que tiverem exatamente a mesma posição na coordenada
$z$, elas poderão ser desenhadas em qualquer ordem.

O ponteiro que aponta para a lista ordenada de interfaces será este:

\iniciocodigo
@<Variáveis Locais (interface.c)@>+=
static struct user_interface **z_list = NULL;
static unsigned z_list_size = 0;
_STATIC_MUTEX_DECLARATION(z_list_mutex);
@
\fimcodigo

Na inicialização de nossa interface, devemos sempre inicializar as
variáveis acima:

\iniciocodigo
@<Inicialização da API de Interface@>+=
MUTEX_INIT(&z_list_mutex);
z_list_size = 0;
z_list = NULL;
@
\fimcodigo

Ao finalizar a nossa API, se a lista de interfaces foi alocada e
portanto tem um valor diferente de nulo, precisamos desalocar e
ajustar o valor para nulo novamente:

\iniciocodigo
@<Finalização da API de Interface@>+=
MUTEX_DESTROY(&z_list_mutex);
if(z_list != NULL && permanent_free != NULL)
  permanent_free(z_list);
z_list = NULL;
z_list_size = 0;
@
\fimcodigo

Da mesma forma que na finalização, quando restauramos o histórico de
interfaces, removendo uma marcação e todas as interfaces depois dela,
teremos que reiniciar nossa lista deixando-a vazia novamente.

\iniciocodigo
@<Restauração de Histórico@>=
MUTEX_WAIT(&z_list_mutex);
if(z_list != NULL && permanent_free != NULL)
  permanent_free(z_list);
z_list = NULL;
z_list_size = 0;
MUTEX_SIGNAL(&z_list_mutex);
@
\fimcodigo

A inicialização desta lista ordenada de interfaces se dará na função
de renderização \monoespaco{\_Wrender\_interface}. Dentro desta
função, deveremos checar se o tamanho desta lista é igual ao número de
interfaces ativas no momento (que pode ser conferida checando a última
marcação de histórico). Se for um valor diferente, isso significa que
interfaces novas foram criadas e devemos gerar novamente nossa lista
ordenada com as interfaces que temos. O código que faz isso é mostrado
abaixo:

\iniciocodigo
@<Gerar Lista Ordenada de Interfaces@>=
if(z_list_size != last_marking -> number_of_interfaces){
  void *p;
  unsigned i, j;
  MUTEX_WAIT(&z_list_mutex);
  // Realocando
  if(z_list != NULL && permanent_free != NULL)
    permanent_free(z_list);
  z_list_size = last_marking -> number_of_interfaces;
  z_list = (struct user_interface **)
             permanent_alloc(sizeof(struct user_interface *) * z_list_size);
  // Copiando para lista:
  p = last_marking -> next;
  for(i = 0; i < z_list_size; i ++){
    if(((struct user_interface *) p) -> type == TYPE_INTERFACE)
      z_list[i] = (struct user_interface *) p;
    else if(((struct user_interface *) p) -> type == TYPE_LINK)
      z_list[i] = ((struct link *) p) -> linked_interface;
    else if(((struct user_interface *) p) -> type == TYPE_SHADER)
      i --; // Não é uma interface
    p = ((struct user_interface *) p) -> next;
  }
  // Ordenando lista (insertion sort):
  for(i = 1; i < z_list_size; i ++){
    j = i;
    while(j > 0 && z_list[j - 1] -> z > z_list[j] -> z){
      p = z_list[j];
      z_list[j] = z_list[j - 1];
      z_list[j - 1] = (struct user_interface *) p;
      j = j - 1;
    }
  }
  MUTEX_SIGNAL(&z_list_mutex);
}
@
\fimcodigo

Note que este cenário no qual a lista deve ser reconstruída é para ser
incomum. Tipicamente ele irá ocorrer na primeira vez que formos
renderizar após o programa começar a execução e após restaurarmos
histórico. O único cenário no qual a execução deste código será mais
frequente é quando novas interfaces forem sendo criadas durante o laço
principal. Contudo, esta é uma má prática e por isso iremos assuir que
somente raramente o código acima entrará em execução para inicializar
a lista. No caso típico, já temos uma lista ordenada com todas as
interfaces, e de vez em quando ajustamos a posição de um dos elementos
se ele for movido.

Uma vez que tenhamos a lista ordenada, basta renderizarmos
individualmente as interfaces na ordem que elas aparecem. Para cada
uma delas devemos carregar o shader correto, passar os atributos,
uniformes e variantes para ele e pedir para renderizar os vértices
compartilhados por todas as interfaces.

Para renderizar as interfaces, invoca-se a função de
API \monoespaco{\_Wrender\_interface}. A função recebe como parâmetro
o tempo atual em microsegundos. Mas para aber quanto tempo se passou
entre uma renderização e outra, precisamos armazenar a nossa última
medida de tempo. Faremos isso na seguinte variável que começará
valendo zero quando nenhuma medida de tempo foi feita:

\iniciocodigo
@<Variáveis Locais (interface.c)@>=
static unsigned long long previous_time = 0;
@
\fimcodigo

A variável sempre precisa ser inicializada para este valor na
inicialização da API. Ou podemos obter valores incorretos se a API for
finalizada e inicializada novamente:

\iniciocodigo
@<Inicialização da API de Interface@>+=
previous_time = 0;
@
\fimcodigo

E eis abaixo nossa função que itera sobre a lista ordenada de
interface e renderiza cada uma delas, atualizando também o tempo
medido. Ela começa primeiro atualizando o tempo, depois carrega os
vértices adequados que representam as interfaces indicando como ele
está representado e passa para uma iteração sobre cada uma das
interfaces ordenadas.

Para cada uma delas, as informações adequadas presentes na estrutura
de interface são passadas para cada uniforme e variante do
shader. Depois de fazer isso, para saber qual textura renderizar se
houver mais de uma, checamos se estamos diante de uma interface com
texturas animadas (comos eria no caso de um GIF animado). Se for o
caso, com base no tempo passado entre esta e a invocação anterior,
atualizamos a contagem de tempo dentro da interface e verificamos se
temos que atualizar ou não o frame.Por fim, renderizamos a interface
atual e sua textura. Depois de executar o laço inteiro, memorizamos o
tempo atual como sendo o tempo anterior para a próxima execução.

A implementação do que foi descrito acima é:

\iniciocodigo
@<Definição de Funções da API (interface.c)@>+=
void _Wrender_interface(unsigned long long time){
  @<Gerar Lista Ordenada de Interfaces@>
  {
    unsigned i, elapsed_time;
    if(previous_time != 0)
      elapsed_time = (int) (time - previous_time);
    else
      elapsed_time = 0;
    // Carregando os vértices do VBO:
    glBindBuffer(GL_ARRAY_BUFFER, interface_vbo);
    // Especificando como os dados estão representados no VBO:
    glVertexAttribPointer(0, 3, GL_FLOAT, GL_FALSE, 5 * sizeof(float),
                          (void *) 0);
    glVertexAttribPointer(1, 2, GL_FLOAT, GL_FALSE, 5 * sizeof(float),
                          (void *) (3 * sizeof(float)));
    glEnableVertexAttribArray(0);
    glEnableVertexAttribArray(1);
    MUTEX_WAIT(&z_list_mutex);
    for(i = 0; i < z_list_size; i ++){
      if(!(z_list[i] -> _loaded_texture) || !(z_list[i] -> visible))
        continue;
      // Escolhendo o shader certo:
      glUseProgram(z_list[i] -> shader_program -> program);
      // Passando os Uniformes:
      glUniform4fv(z_list[i] -> shader_program -> uniform_foreground_color, 4,
                   z_list[i] -> foreground_color);
      glUniform4fv(z_list[i] -> shader_program -> uniform_background_color, 4,
                   z_list[i] -> background_color);
      glUniformMatrix4fv(z_list[i] -> shader_program ->
                           uniform_model_view_matrix, 1, false,
                         z_list[i] -> _transform_matrix);
      glUniform2f(z_list[i] -> shader_program -> uniform_interface_size,
                  z_list[i] -> width, z_list[i] -> height);
      glUniform2f(z_list[i] -> shader_program -> uniform_mouse_coordinate,
                  z_list[i] -> mouse_x, z_list[i] -> mouse_y);
      // O shader recebe contagem de tempo em segundos módulo 1 hora
      glUniform1f(z_list[i] -> shader_program -> uniform_time,
                  ((double) (time % 3600000000ull)) / (double) (1000000.0)); 
      glUniform1i(z_list[i] -> shader_program -> uniform_integer,
                 z_list[i] -> integer);
      // Animating texture
      if(z_list[i] -> animate && z_list[i] -> number_of_frames > 1 &&
         z_list[i] -> max_repetition != 0){
        z_list[i] -> _t += elapsed_time;
        z_list[i] -> current_frame %= z_list[i] -> number_of_frames;
        while(z_list[i] -> _t >=
                     z_list[i] -> frame_duration[z_list[i] -> current_frame]){
          z_list[i] -> _t -=
            z_list[i] -> frame_duration[z_list[i] -> current_frame];
          z_list[i] -> current_frame ++;
          z_list[i] -> current_frame %= z_list[i] -> number_of_frames;
        }
      }
      // Rendering:
      if(z_list[i] -> _texture1 != NULL)
        glBindTexture(GL_TEXTURE_2D,
                      z_list[i] -> _texture1[z_list[i] -> current_frame]);
      else
        glBindTexture(GL_TEXTURE_2D, default_texture);
      glDrawArrays(GL_TRIANGLE_FAN, 0, 4);
    }
    MUTEX_SIGNAL(&z_list_mutex);
    glBindTexture(GL_TEXTURE_2D, 0);
  }
  previous_time = time;
}
@
\fimcodigo

\secao{8. Movendo, Rotacionando e Redimencionando Interfaces}

\subsecao{8.1. Movendo Interfaces}

Mover uma interface significa atualizar suas variávels $(x, y,
z)$. Mas além disso, precisamos atualizar também a matriz de
modelo-visualização para que tenha novos valores $x$ e $y$, além de
que se acoordenada $z$ mudou, então podemos ter que mudar a posição da
interface com outra na lista ordenada de interfaces que usamos para
determinar qual desenhamos primeiro. Para tudo isso, precisamos também
reservar o uso do Mutex da interface para garantir que duas invocções
da função não se sobreponham.

A função que move interfaces em nossa API
é \monoespaco{\_Wmove\_interface} e definimos ela assim:

\iniciocodigo
@<Definição de Funções da API (interface.c)@>+=
void _Wmove_interface(struct user_interface *i,
                      float new_x, float new_y, float new_z){
  GLfloat x, y;
  GLfloat cos_theta = cos(i -> _rotation);
  GLfloat sin_theta = sin(i -> _rotation);
  MUTEX_WAIT(&(i -> mutex));
  i -> x = i -> _x = new_x;
  i -> y = i -> _y = new_y;
#if defined(W_FORCE_LANDSCAPE)
  if(*window_height > *window_width){
     i -> _x = *window_width - new_y;
     i -> _y = new_x;
  }
#endif
  @<Conversão de Coordenadas e Tamanho@>
  i -> _transform_matrix[12] = x +
    (i -> height / (*window_width)) * sin_theta -
    (i -> width / (*window_width)) * cos_theta;
  i -> _transform_matrix[13] = y -
    (i -> width / (*window_height)) * sin_theta -
    (i -> height / (*window_height)) * cos_theta;
  if(new_z != i -> z){ // Atualizando lista ordenada de interfaces
    unsigned j;
    i -> z = new_z;
    MUTEX_WAIT(&z_list_mutex);
    for(j = 0; j < z_list_size; j ++){
      if(z_list[j] == i){
        while(j > 0 && i -> z < z_list[j - 1] -> z){
          z_list[j] = z_list[j - 1];
          z_list[j - 1] = i;
          j --;
        }
        while(j < z_list_size - 1 && i -> z > z_list[j + 1] -> z){
          z_list[j] = z_list[j + 1];
          z_list[j + 1] = i;        
          j ++;
        }
      }
    }
    MUTEX_SIGNAL(&z_list_mutex);
  }
  MUTEX_SIGNAL(&(i -> mutex));
}
@
\fimcodigo

\subsecao{8.2. Rotacionando Interfaces}

Rotacionar interfaces envolve apenas reservar o mutex da interface,
atualizar sua variável de rotação e também atualizar sua matriz de
modelo-visualização. O código para isso segue abaixo:

\iniciocodigo
@<Definição de Funções da API (interface.c)@>+=
void _Wrotate_interface(struct user_interface *i, float rotation){
  GLfloat x, y;
  GLfloat cos_theta = cos(rotation);
  GLfloat sin_theta = sin(rotation);
  MUTEX_WAIT(&(i -> mutex));
  i -> rotation = i -> _rotation = rotation;
#if defined(W_FORCE_LANDSCAPE)
  if(*window_height > *window_width)
    i -> _rotation += M_PI_2;
#endif 
  @<Conversão de Coordenadas e Tamanho@>
  i -> _transform_matrix[0] = (2 * i -> width / (*window_width)) *
    cos_theta;
  i -> _transform_matrix[1] = (2 * i -> width / (*window_height)) *
    sin_theta;
  i -> _transform_matrix[4] = -(2 * i -> height / (*window_width)) *
     sin_theta;
  i -> _transform_matrix[5] = (2 * i -> height / (*window_height)) *
    cos_theta;
  i -> _transform_matrix[12] = x +
    (i -> height / (*window_width)) * sin_theta -
    (i -> width / (*window_width)) * cos_theta;
  i -> _transform_matrix[13] = y -
    (i -> width / (*window_height)) * sin_theta -
    (i -> height / (*window_height)) * cos_theta;
  MUTEX_SIGNAL(&(i -> mutex));
}
@
\fimcodigo

\subsecao{8.3. Redimensionando Interfaces}

Redimensionar interfaces, assim como rotacioná-las, envolve apenas
reservar um mutex, atualizar as variáveis de tamanho e atualizar a
matriz de modelo-visualização. Entretanto, também devemos tomar
cuidado com as variáveis de tamanho máximo e mínimo da interface,
lembrando que um tamanho máximo ou mínimo não-positivo significa que
tal limite não existe.

A função que redimenciona interfaces é:

\iniciocodigo
@<Definição de Funções da API (interface.c)@>+=
void _Wresize_interface(struct user_interface *i,
                        float new_width, float new_height){
  GLfloat x, y;
  GLfloat cos_theta = cos(i -> _rotation);
  GLfloat sin_theta = sin(i -> _rotation);
  MUTEX_WAIT(&(i -> mutex));
  i -> width = new_width;
  i -> height = new_height;
  @<Conversão de Coordenadas e Tamanho@>
  i -> _transform_matrix[0] = (2 * i -> width / (*window_width)) *
    cos_theta;
  i -> _transform_matrix[1] = (2 * i -> width / (*window_height)) *
    sin_theta;
  i -> _transform_matrix[4] = -(2 * i -> height / (*window_width)) *
     sin_theta;
  i -> _transform_matrix[5] = (2 * i -> height / (*window_height)) *
    cos_theta;
  i -> _transform_matrix[12] = x +
    (i -> height / (*window_width)) * sin_theta -
    (i -> width / (*window_width)) * cos_theta;
  i -> _transform_matrix[13] = y -
    (i -> width / (*window_height)) * sin_theta -
    (i -> height / (*window_height)) * cos_theta;
  MUTEX_SIGNAL(&(i -> mutex));
}
@
\fimcodigo

\secao{8. Interagindo com Interfaces}

Dadas as interfaces ativas, pode-se interagir com elas passando o
cursor do mouse delas, clicando com o botão direito do mouse nelas,
clicando com o esquerdo ou clicando com o botão do meio. Para
gerenciar isso usamos a função \monoespaco{\_Winteract\_interface} que
nos informa o que o mouse está fazendo e automaticamente executa as
funções associadas com cada interação para cada interface.

Para isso ser possível, temos que memorizar o estado anterior dos
botões do mouse, não apenas lidar com o estado atual. Por causa disso,
vamos usar as seguintes variáveis estáticas para memorizar o último
estado dos botões:

\iniciocodigo
@<Variáveis Locais (interface.c)@>=
static bool mouse_last_left_click = false, mouse_last_middle_click = false,
  mouse_last_right_click = false;
@
\fimcodigo

Inicializamos tais variáveis com o valor falso na inicialização de
nossa API:

\iniciocodigo
@<Inicialização da API de Interface@>+=
mouse_last_left_click = false;
mouse_last_middle_click = false;
mouse_last_right_click = false;
@
\fimcodigo

Sabendo o estado anterior do mouse e o estado atual, somos capazes de
dizer não apenas se um botão está sendo pressionado, mas também se ele
começou a ser pressionado agora ou a pressão sobre ele está apenas
continuando. Saberemos não apenas que um botão não está sendo
pressionado, mas também se ele foi solto em algum momento entre o
frame anterior e o atual. Só assim seremos capazes de executar
corretamente as funções de interação de interface.

Em um dado frame, só podemos interagir com uma única interface, dado
que só temos um cursor do mouse. Mas se mais de uma interface ocupar a
mesma posição, como saberemos com qual interagir. Ora, felizmente
armazenamos em uma lista ordenada as interfaces na ordem em que elas
são desenhadas na tela. Devemos então priorizar a interface que é
desenhada depois, pois é ela que o usuário está vendo sob o cursor do
mouse.

Então, para interagir com interfaces, percorremos em ordem reversa a
lista ordenada de interfaces até acharmos aquela que está diretamente
sob o mouse. É com ela que iremos interagir. Só temos que também ao
percorrer a lista, marcar que o mouse não está sobre qualquer uma
outra. E se for o caso, executar a função \monoespaco{on\_mouse\_out} da
interface. Só então, checamos a interação com a interface certa e
executamos a função adequada se existir.

O código da função que faz isso é:

\iniciocodigo
@<Definição de Funções da API (interface.c)@>+=
void _Winteract_interface(int mouse_x, int mouse_y, bool left_click,
                          bool middle_click, bool right_click){
  int i;
  struct user_interface *previous = NULL, *current = NULL;
  MUTEX_WAIT(&z_list_mutex);
  for(i = z_list_size - 1; i >= 0; i --){
    float x, y;
    @<Converter Coordenadas do Mouse para x e y@>
    z_list[i] -> mouse_x = x - z_list[i] -> x + (z_list[i] -> width / 2);
    z_list[i] -> mouse_y = y - z_list[i] -> y + (z_list[i] -> height / 2);
    if(current == NULL &&
       z_list[i] -> mouse_x  > 0 && z_list[i] -> mouse_x < z_list[i] -> width &&
       z_list[i] -> mouse_y  > 0 && z_list[i] -> mouse_y < z_list[i] -> height)
      current = z_list[i];
    else{
      if(z_list[i] -> _mouse_over){
         z_list[i] -> _mouse_over = false;
         previous = z_list[i];
       }
    }
  }
  MUTEX_SIGNAL(&z_list_mutex);
  if(previous != NULL && previous -> on_mouse_out != NULL){
    previous -> on_mouse_out(previous);
  }
  if(current != NULL){
    if(current -> _mouse_over == false){
      current -> _mouse_over = true;
      if(current -> on_mouse_over != NULL)
        current -> on_mouse_over(current);
    }
    if(left_click && !mouse_last_left_click && current -> on_mouse_left_down)
      current -> on_mouse_left_down(current);
    else if(!left_click && mouse_last_left_click && current -> on_mouse_left_up)
      current -> on_mouse_left_up(current);
    if(middle_click && !mouse_last_middle_click &&
       current -> on_mouse_middle_down)
      current -> on_mouse_middle_down(current);
    else if(!middle_click && mouse_last_middle_click &&
            current -> on_mouse_middle_up)
      current -> on_mouse_middle_up(current);
    if(right_click && !mouse_last_right_click && current -> on_mouse_right_down)
      current -> on_mouse_right_down(current);
    else if(!right_click && mouse_last_right_click &&
            current -> on_mouse_right_up)
      current -> on_mouse_right_up(current);
  }
  mouse_last_left_click = left_click;
  mouse_last_middle_click = middle_click;
  mouse_last_right_click = right_click;
}
@
\fimcodigo

A parte que omitimos acima é como obtemos a coordenada $(x, y)$ que
usamos para comparar com a posição de cada interface à partir das
coordenadas de mouse que recebemos como argumento. A coordenada $(x,
y)$, muitos casos é exatamente a mesma coordenada de mouse que
recebemos como argumento. Mas se a interface estiver rotacionada,
então precisamos fazer um ajuste para poder medir corretamente.

Se o mouse estiver bem na ponta inferior direita de interface
retangular, e subitamente ela for rotacionada um pouco, a interface
não estará mais sob o mouse. Para levar em conta esse tipo de rotação,
a forma mais fácil é ignorarmos que a interface está rotacionada e
rotacionarmos as coordenadas do mouse na direção oposta da rotação da
interface e usando o centro da interface como eixo de rotação. Se a
coordenada transformada ainda estiver dentro dos limites retangulares
da interface, isso significa que o mouse realmente está sobre a nossa
interface.

O código que gera a coordenada $(x, y)$ então apenas checa se a
interface está rotacionada. Se não estiver, usamos como valores as
próprias coordenadas do mouse, sem precisar fazer cálculos adicionais
de conversão. Se estiver, então obtemos $(x, y)$ fazendo a rotação
descrita na coordenada do mouse:

\iniciocodigo
@<Converter Coordenadas do Mouse para x e y@>=
if(z_list[i] -> rotation == 0.0){
  x = mouse_x;
  y = mouse_y;
}
else{
 float cos_theta = cos(- (z_list[i] -> rotation));
 float sin_theta = sin(- (z_list[i] -> rotation));
 x = (mouse_x - z_list[i] -> x) * cos_theta -
       (mouse_y - z_list[i] -> y) * sin_theta;
 y = (mouse_x - z_list[i] -> x) * sin_theta +
       (mouse_y - z_list[i] -> y) * cos_theta;
 x +=  z_list[i] -> x;
 y +=  z_list[i] -> y;
}
@
\fimcodigo


%%%%%%%%%%%%%%%%%%%%%%%%%%%%%%%%%%%%%

\secao{Referências}

\referencia{Knuth, D. E. (1984) ``Literate Programming'', The Computer
  Journal, Volume 27, Edição 2, Páginas 97--111.}


\fim
